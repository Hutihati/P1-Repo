\section{Konklusion}
Der kan konkluderes WPT's potentiale udvides ved brug at en magnetisk induktiv kobling med resonans frekvens fremfor brug af en standart magnetisk induktiv kobling. Resonans kobling har en klar fordel i henholdsvis til effekt over afstand, vist ved forsøg med Ikea oplader, hvor der er et stort tab når afstandens vokser, og ved en afstand på kun 0,5cm er nyttevirkningen faldet til 44\%. Dette er selvfølgelig kun relevant for den given oplader, hvor andre oplader der bruger samme princip kan op nå en effekt på 90\% ved en afstand på op til 10cm. Men ind forhold til en oplader med resonans frekvens hvor der kan opnåets en effekt 90\% ved op til en meter. Dog er det sværre at intrigere i små enheder, idet at der skal intrigeres en kapasitor i modtageren og afsender. Resonans har også den fordel at flere modtager kan kobles op til samme modtager uden bemærkelsesværdig tab, så længe det er samme resonans frekvens. Ved brug af resonans frekvens kan der også opnåets et højt energi output ved høj frekvens, men der kan kun være lille udsving i frekvens for at der ikke opnås tab i overførelsen. Det kan konkluderes at resonans frekvens er fremtiden inde WPT nå afstand og output er gejlende. 