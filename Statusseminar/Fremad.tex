\chapter{PV-fremadrettet}
Nu da vi har lavet vores første forsøg, hvor vi har set på, hvordan spændingen skifter over en modstand, kapacitator og spole ved ændring på frekvensen, kan vi nu gå videre til vores forsøg om WPT. Vi fik chancen for at omlægge den teori, vi har læst om, ud til en forsøgsopstilling, hvor vi kunne aflæse forskellige resultater, der giver mulighed for videre arbejde. Det har ikke lokket os fra at skulle arbejde videre med projektet, snarere det modsatte. Vi har fået nyt blod på tanden og glæder os til de afsluttende forsøg. Forsøgene skal benyttes til databehandlingen, så vi kan sammenligne resultaterne, så vi har konkrete tal at arbejde ud fra, når vi skal videre med problemløsningen. Vores strategi for fremtiden er at vi vil skriftes til at lave forsøg og dobbelttjekke hinandens forsøgsopstillinger. Når den ene del af gruppen er i laboratorierne vil den anden del af gruppen ihærdigt skriver videre, laver beregninger, skematics osv. til rapporten. Det er vores målsætning at holde møde hver fredag, så vi kan holde trit med vores tidsplan, og at hvis vi kommer bag ud, er det muligt for de andre i gruppen at træde til og hjælpe et medlem. Dette er ingen individuel rapport, og vi skal alle sammen kunne stå inde for rapporten. Ergo er det vigtigt, at man hjælper hinanden, for hvis der er en som taber, taber hele gruppen. Det er lige meget hvis skyld, det er, for vi er alle i samme båd.

For at holde gejsten oppe i gruppen sørger vi for at holde hinanden i nakken, så der ikke er nogen, der får problemer og falder bagud i forhold til gruppen. Projektet er en samlet indsats, så det er vigtigt at få lavet opsamlinger i gruppen, så alle kan følge med, og alle har samme udgangspunkt for senere arbejde. Derudover diskuterer vi om, hvad der interesserer os ved projektet, og hvad vi gerne vil have ud af det færdige produkt, så det holder os fast på et endeligt mål. Hvis vi arbejder med det, vi finder interesse for, er det lettere at holde sig fokuseret på arbejdsopgaverne, og man får lyst til at gøre en god indsats (ikke kun for en selv, men også for gruppen som helhed).

Som gruppe bliver vi enige om, hvilke arbejdsområder vi skal undersøge og skrive om. Herefter inddeler vi os i mindre grupper for hver arbejdsopgave (2-3 mand pr. gruppe). Dette gør, at vi hurtigt kan få indsamlet brugbar viden og udført en stort stykke arbejde, men at vi samtidig ikke står alene med en opgave. Det at være i små grupper gør også, at man kan få flere input og idéer til, hvad man eventuelt kan bringe ind over opgaven, hvilket gør at projektet bliver mere nuanceret. For at holde styr på, hvor langt hver gruppe er, og hvad de hver især har skrevet, så holder vi jævnligt møder til at opsummere processen.

Da vi er igennem problemanalysen, så har vi fået dannet os en god grundviden om projektet, som vi videre kan benytte til forsøg, modeller og projektløsningen senere hen. Dette betyder også, at vi nu skal til at specificere os på enkelte dele af projektet, så vi får indsnævret vores undersøgelser.
