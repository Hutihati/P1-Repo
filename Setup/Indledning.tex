\chapter{Indledning}
Strøm er noget, alle kender til. Det bliver brugt overalt, da næsten alle vores apparater benytter strøm i dag. Samfundet er derfor meget afhængig af at kunne sende strømmen ud til forbrugerne. Dette foregår via et ledningsnetværk, hvor ledningerne enten kan være kabler, som er kravet i jorden, eller så kan det være luftledninger, som er ophængt i master. "(Kilde Gyldendal)" Luftledninger og kabler i jorden er den bedste teknologi, som findes til at sende elektricitet ud til forbrugerene, det er her Wireless Power Transfer (WPT), altså trådløs energioverførelse, kunne være en rigtig god teknologi. Ved at benytte WPT så bliver samfundet fri for at lægge jordkabler eller ophænge luftledninger, men i stedet overføre energien igennem luften og hen til produktet kun ved hjælp af en transmitter og en modtager.

Wireless Power Transfer teknologien, som den er i dag, har mange problemstillinger, som gør, at det ikke er muligt, at kunne sende elektricitet på længere afstande. Der findes flere forskellige måder at kunne sende energi trådløst på. En metode er Mikrobølger, hvor en anden metode er induktiv elektromagnetisme. Ud fra disse problemstillinger, så er teknologien ikke længere end ved trådløs opladning af mobiltelefoner. Ved at lægge mobilen direkte ovenpå afsenderen, så afstanden, som den trådløse energi skal løbe, er mindst mulig, så er det muligt, at kunne oplade mobilen trådløst.

Tilbage i 1890 begyndte fysikeren Nikola Tesla (1856-1943) at udarbejde teorier og forsøg omkring at kunne overføre elektricitet over afstande uden brug af ledninger. Nikola Tesla fokuserede på, at kunne sende den trådløse energi, ved brug af elektromagnetisme. "(kilde Historie)". Der bliver hele tiden bygget videre på Tesla's teorier og forsøg. Induktiv kobling og resonant frekvens er hovedpunkter for at kunne forbedre trådløs energioverførelse. "(Kilde MIT)".

For at teknologien skal kunne forbedres, så tager rapporten udgangspunkt i resonant frekvens. Dette gøres ved hjælp af at se på en trådløs oplader til en mobil, og grundlæggende kredsløb for overførsel af energi.

Derved leder dette ind til et initierende problem:

\textit{Hvordan kan afstande på trådløs energioverførelse forbedres?}