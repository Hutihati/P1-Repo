\chapter{indledning}
Samfundet vi liver i dag er yderst afhængig af elektricitet for at kunne funger og vi som samfund er blevet var til den afhængighed, hvilke har gennem tiden ført til ny og bedre teknologi, og derved er ideen omkring trådløs energi overførelse også opstået. Trådløs energi overførelse er noget rimeligt nyt, der ses i hverdagen, men teknologien stammer tilbage fra 1890´erne hvor Nikola Tesla havde allerede forsøgte sig med, at skabe og sende trådløs elektricitet, (Bellow,2016) og drømte om at sende energien igennem den øvre atmosfære. Nikola Tesla drømte om fri energi til hele verden ikke blot kun til en by, men der er stadig lang vej den dag i dag. Metoder der bliver brugt i dag har stadig utrolig mange problemstillinger før det vil være muligt og erstatte traditionelle kabler. Før det er muligt, at lade trådløs elektricitet overtage hverdagen, er der nogle krav til denne teknologi 

Der er mange forskellige måder og sende energien på, men et af kravene der er meget vigtige er, at disse metoder ikke gør skade på mennesket, ikke mindst med det er det vigtig at energi kan blive sendt over længere distancer uden for meget spild og ustabil forbindelse. I dag er der meget fokus på global opvarmning og ikke mindst grøn energi, hvor Danmark har 2020 og 2050 planerne derfor er det også vigtigt, at der er tænkt miljøbevist før teknologien ville kunne slå igennem og blive den mest anvendte. 

Der er to forskellige tilgange til denne teknologi, WSP (Wireless Power Transfer). En af tilgangene benytter sig af højfrekvens bølger, som mikrobølger eller lasere, en af disse sendes gennem luften hen til en modtager der kan omdanne den modtaget stråle/mikrobølge energien (photonerne) til elektricitet igen. Hvis denne metode anvendes er det muligt, at sende energi over længere afstande uden nogen form for fysik tilkobling, dog er det hovedsaglige problem med denne teknologi, at hvis der kommer noget imellem afsenderen og modtagerne ophøre overførelsen, ikke desto mindre kan det være skadeligt for mennesker, at udsætte dem for strålerne. Den anden tilgangsmåde forholder sig lidt anderledes her anvendes der elektromagnetisme, denne tilgangsmåde benytter sig af de elektroner der løber gennem en ledning, når elektroner løber gennem en ledning skabes der et magnetisk felt omkring (Amperes lov).  Når et magnetisk felt får indflydelse på en ledning skubber det magnetiske felt  til elektronerne, som skaber elektricitet (Faraday´s lov). I sådan et system har i forhold til den anden tilgang en begrænset rækkevidde, men den er alt mere sikker at anvende for mennesker. Denne teknologi bliver i dag brugt til flere forskellige produkter, eksempelvis bliver det brugt til og lade elektriske tandbørster op med og ikke mindst til pacemakers. Dette er denne type teknologi projektet vil fokuser på. 
