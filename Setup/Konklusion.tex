\chapter{Konklusion}
Der kan konkluderes, at WPT's potentiale udvides ved brug at en magnetisk induktiv kobling med resonant frekvens fremfor brug af en standart magnetisk induktiv kobling. Resonant kobling har en klar fordel i forhold til effekt over afstand, vist ved forsøg med IKEA oplader, hvor der er et stort tab, når afstanden vokser, og ved en afstand på kun $0,5 \, cm$ er nyttevirkningen faldet til $44\%$. Dette er selvfølgelig kun relevant for den given oplader, hvor andre oplader, der bruger samme princip, kan nå op på en effekt på $90\%$ ved en afstand på op til $10 \, cm$, i forhold til en oplader med resonant frekvens hvor der kan opnåets en effekt $90\%$ ved op til en meter. Dog er det sværre at intrigere i små enheder, idet der skal intrigeres en kapasitor i modtageren og afsender. Resonans har også den fordel at flere modtager kan kobles op til samme afsender uden bemærkelsesværdig tab, så længe det er samme resonante frekvens. Ved brug af resonant frekvens kan der også opnåets et højt energi output ved høj frekvens, men der kan kun være lille udsving i frekvens, for at der ikke opnås tab i overførelsen. Det kan konkluderes, at resonant frekvens er fremtiden inden for WPT, når afstand og output er gældende.