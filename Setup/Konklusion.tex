\chapter{Konklusion}

Der kan konkluderes, at WPT's potentiale udvides bedst ved brug af en magnetisk induktiv kobling med resonant frekvens. Resonant kobling har en klar fordel i forhold til effekt over afstand, vist ved forsøg med IKEA oplader, hvor der er et stort tab, når afstanden vokser, og ved en afstand på kun $0,5 \, cm$ er nyttevirkningen faldet til $44\%$. Dette er selvfølgelig kun relevant for den given oplader, hvor andre oplader, der bruger samme princip, kan nå op på en effekt på $90\%$ ved en afstand på op til $10 \, cm$, i forhold til en oplader med resonant frekvens hvor der kan opnås en effekt $90\%$ ved op til en meter. Dog er det sværre at intrigere i små enheder, idet der skal intrigeres en kapasitor i modtageren og afsender. Resonant frekvens har også den fordel, at flere modtagere kan kobles op til samme afsender uden bemærkelsesværdig tab, så længe der er samme resonant frekvens. Ved brug af resonant frekvens kan der også opnås et højt energi output, men der kan kun være lille udsving i frekvens, før der sker for store tab i overførelsen. Det kan konkluderes, at resonant frekvens er fremtiden inden for WPT, når afstand og output er gældende.