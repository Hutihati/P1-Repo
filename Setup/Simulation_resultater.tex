Ud fra målingerne over LCR-kredsløbet kan strømstyrken for systemet beregnes ud fra en given frekvens, som foretaget tidligere. Herefter kan beregningerne verificeres ved at holdes op mod angivne data fra en simulation af samme kredsløb.

Resultaterne fra simulationen kan også benyttes til at vise udviklingen for strømstyrken ved ændring af frekvensen. Derudover kan der ændres på størrelsen for modstanden på resistoren i systemet for at se, hvordan strømstyrken ændre sit udsving omkring den resonante frekvens.

\begin{figure}[H]
\includegraphics[scale=0.5]{Vildledning/Schematics/Graf_forsøg1}
\end{figure}

Graf X viser tre funktioner for strømstyrken ved en stigende frekvens. Forskellen på de tre sæt data for funktionerne er, at de henholdsvis har indsat en modsat på $250 \Omega$, $500 \Omega$ eller $1000 \Omega$. Funktionen med modstanden på $250 \Omega$ giver et stort udslag i løbet af en stigning i frekvens mellem $1 kHz$ og $2 kHz$. Toppunktet for funktionen er skarpt optegnet ved den resonante frekvens, hvorefter strømstyrken er blevet mere end halveret ved en frekvens på $3 kHz$. På højere frekvenser falder strømstyrken fortsat, men faldet stilner roligt af. Ved funktionen med en modstand på $500 \omega$ er udsvinget for strømstyrken ved den resonante frekvens halveret i forhold til første funktion. Til gengæld sker faldet ved større frekvenser ikke så drastisk. Derved bliver frekvensspektret forstørret for et fornuftigt strømstyrkeoutput. Ved den sidste funktion med en modstand på $1000 \Omega$ er udslaget ved den resonante frekvens næste jævnet ud sammenlignet med de to andre funktion. Til gengæld er faldet for strømstyrken lille, og frekvensområdet for en stabil strømstyrke er forlænget meget i forhold til de første to funktioner. Sammenlignet med første funktion, så er det overordnede fald i strømstyrken lav for tredje graf. Hvor den første graf med en modstand på $250 \Omega$ falder med op til 75 procent fra den resonante frekvens på $2.197 kHz$ til $4 kHz$, så sker der kun et fald på 30 procent for den tredje funktion med en modstand på $1000 \Omega$ over samme frekvensområde.