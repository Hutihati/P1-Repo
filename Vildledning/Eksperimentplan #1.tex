\documentclass[a4paper,11pt]{memoir}
\usepackage{mathtools}
\usepackage{graphicx}
\usepackage[utf8]{inputenc}
\usepackage{float}
\begin{document}
\chapter{Eksperimentplan 1}
P1 - Gruppe C-16a - 07-Nov-16
\newpage

\section{Hvad vil vi opstille?}
\begin{itemize}
\item R-kreds
\item CR-kreds
\item LCR-kreds
\end{itemize}
Til alle opstillingerne vil der i starten benyttes et oscilloskop som kilde, sat til vekselspænding ved 5 V, 50Hz.

De tre opsætninger benyttes som reference punkter til videre eksperimentring.  
\section{R-kreds}
Til at starte med vil vi opstille, det måske aller simpleste kredsløb, ved bare at sende en vekselspænding over en resistor og måle spændingsforskellen og strømmen over det.

Herefter vha. $U=R\cdot I$ kan der så undersøges om den aflæste værdi af resistoren er den samme som den målte.
\section{CR-kreds}
Opstillingen her er ens med R-kredsen, uden ampere-meter, bare at der nu er sat en kapacitator på og volt-meteret er flyttet hen over kapacitatoren.

Herefter ses der på om der er sket en ændring i spændingsforskellen over kapacitatoren, sammenlignet med den mængde spænding der er tilført systemet.
\section{LCR-kreds}
Dette er den vigtigste kreds der undersøges, da der nu er lavet et loop ved at sætte en spole i kredsløbet. Volt-meteret der benyttes her skulle gerne være et der kan tegne grafer, specifikt $(U,t)$.

Kredsløbet er en forlængelse af de to andre. Volt-meteret er nu bare flyttet til hen over spolen, da dette er den komponent med størst relevans.

Her undersøges der den spændingskurve tegnet over spolen. Denne kan herefter så sammenlignes med kurven i Plecs, hvis det er muligt at komme tæt på de virkelige værdier i programmet, så man får en nogenlunde præcis simulation.
\end{document}