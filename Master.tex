\input{Setup/preamble_ida16-Mads.tex}

%% ^^Alt setup findes i preamble^^ %%
\raggedbottom


\begin{document}
\frontmatter %% romertegn som sidetal %%

%% folderinput er så man frit kan flytte filerne ned til de relevante mapper %%
	\begin{folderinput}{Setup}

\thispagestyle{empty}
\begin{flushright}
\vspace{3cm}

\phantom{hul}

\phantom{hul}

\phantom{hul}

\textsl{\Huge Trådløs energioverførsel med fokus på trådløs mobilopladning} \\ \vspace{1cm}

\rule{13cm}{3mm} \\ \vspace{1.5cm}
\vspace{1cm}

%\includegraphics[width=0.864\textwidth]{Vildledning/Schematics/Vinkelflux}

\vspace{2cm} 
\textsc{\Large P1 Projekt \\
Gruppe C2-16a \\
Energi\\
Aalborg Universitet\\
Den 19. december 2016\\}
\end{flushright}

\newpage

\cleardoublepage
% Dette er LaTeX-versionen af titelbladet for TNB studenterrapporter
% Filen kræver:
% Universitetets logo:  AAU-logo-stud-UK eller AAU-logo-stud-DK
% Synopsis: En fil ved navn synopsis.tex

% Udarbejdet af: Jesper Nørgaard (jesper@noergaard.eu) 10. april 2012
% Redigeret af Mathias S. Hansen (mat_elm13@hotmial.com) 9. november 2016

%\phantomsection
\thispagestyle{empty}
%\pdfbookmark[0]{Titelblad}{titelblad}

\begin{minipage}[t]{0.48\textwidth}
\vspace*{-25pt}			%\vspace*{-9pt}
\includegraphics[height=4cm]{billeder/AAU-logo-stud-DK-RGB}
\end{minipage}
\hfill
\begin{minipage}[t]{0.48\textwidth}
{\small 
\textbf{Første Studieår v/ }\\
\textbf{School of Engineering and Science (SES)}  \\
Energi \\
Strandvejen 12-14 \\
9000 Aalborg \\
http://www.tnb.aau.dk}
\end{minipage}

\vspace*{0.2cm}

\begin{minipage}[t]{0.48\textwidth}
\textbf{Titel:} \\[5pt]\bigskip\hspace{2ex}
Trådløs Mobilopladning

\textbf{Projekt:} \\[5pt]\bigskip\hspace{2ex}
P1-projekt

\textbf{Projektperiode:} \\[5pt]\bigskip\hspace{2ex}
September 2016 - December 2016

\textbf{Projektgruppe:} \\[5pt]\bigskip\hspace{2ex}
C2-16a	

\textbf{Deltagere:} \\[5pt]\hspace*{2ex}

\noindent\begin{tabular}{ll}
\makebox[2.5in]{\hrulefill} \\
Daniel Revsbech Pedersen \\
\end{tabular} \\[10pt]\hspace*{2ex}


\noindent\begin{tabular}{ll}
\makebox[2.5in]{\hrulefill} \\
Mads Lindstrøm Paulsen \\
\end{tabular} \\[10pt]\hspace*{2ex}

\noindent\begin{tabular}{ll}
\makebox[2.5in]{\hrulefill} \\
Mathias Stenberg Hansen \\
\end{tabular} \\[10pt]\hspace*{2ex}

\noindent\begin{tabular}{ll}
\makebox[2.5in]{\hrulefill} \\
Nicolai Nørgaard Munk \\
\end{tabular} \\[10pt]\hspace*{2ex}

\noindent\begin{tabular}{ll}
\makebox[2.5in]{\hrulefill} \\
Sisse Sorgenfri Jensen \\
\end{tabular} \\[10pt]\hspace*{2ex}

\noindent\begin{tabular}{ll}
\makebox[2.5in]{\hrulefill} \\
Torben Brund Jørgensen \\
\end{tabular} \\[10pt]\hspace*{2ex}

%Daniel Revsbech Pedersen \\[5pt]\\\hspace*{2ex}
%Julian Bo Larsen \\[5pt]\\\hspace*{2ex}
%Mads Lindstrøm Paulsen \\[5pt]\\\hspace*{2ex}
%Mathias Stenberg Hansen \\[5pt]\\\hspace*{2ex}
%Nicolai Nørgaard Munk \\[5pt]\\\hspace*{2ex}
%Sisse Sorgenfri Jensen \\[5pt]\\\bigskip\hspace{2ex}
%Torben Brund Jørgensen

\textbf{Vejledere:} \\[10pt]\hspace*{2ex}

\noindent\begin{tabular}{ll}
\makebox[2.5in]{\hrulefill} \\
Christian Uhrenfeldt \\
\end{tabular}
%Christian Uhrenfeldt \\\hspace{2ex}%\\\bigskip\hspace{2ex}

%\vspace*{1cm}

\end{minipage}
\hfill
\begin{minipage}[t]{0.483\textwidth}
Synopsis: \\[5pt]
\fbox{\parbox{7cm}{\bigskipThis project aims to inform the reader about the usage of wireless power transfer. The project discusses the use of WPT in the past, the present and a possible future. This project uses mathematical modelling to give the reader an insight in magnetic inductive coupling. This paper is founded on basic knowledge of electromagnetism in addition to wireless power transfer. This knowledge is used in experiments of a LCR-circuit, in order to prove the connection between frequency and voltage, and thus the peak current at resonant frequency. Furthermore, an experiment to show the loss of power at increasing distance between sender and reciever. This paper concludes that resonant magnetic coupling will provide the most efficient energy transfer and will open up for further research on the subject matter for a future of more effecient long range wireless power transfer.\bigskip}}

\vspace*{16.5cm}

\textbf{Oplagstal: ?} \\
\textbf{Sidetal: ?} \\
\textbf{Appendiks: ?} \\ 
\textbf{Afsluttet: DATO}

\end{minipage}

\vfill


%{\footnotesize\itshape Rapportens indhold er frit tilgængeligt, men offentliggørelse (med kildeangivelse) må kun ske efter aftale med forfatterne.}

% Rapportens indhold er frit tilgængeligt, men offentliggørelse (med kildeangivelse) må kun ske efter aftale med forfatterne.
% The content of the report is freely available, but publication (with source reference) may only take place in agreement with the authors.
\cleardoublepage

Forord:

Projektet er udarbejdet af gruppe C2-15a i forbindelse med P1 i perioden 10. oktober til 19. december 2016.

Projektrapporten er skrevet i \LaTeX. Til projektet er der benyttet TI-nspire.cas og Excel 2016 til beregning af data. Derudover har vi benyttet Plex til simulation af vores forsøg.

Vi vil først og fremmest gerne rette en tak til vores vejleder Christian Uhrenfeldt for at assisterer os igennem projektets forløb. Ikke kun med forståelse af projektets oplæg, men også for besvarelse af spørgsmål, vejledning for projektets retning og fokuspunkter, samt idéer til og forståelse af forsøg. Derudover vil vi gerne rette en tak til Flemming Kristoffersen, som har haft interesse for og været behjælpelig med gennemgangen af forsøg igennem projektet. Til slut retter vi en tak til Esben Skovsen for hans undervisning omkring elektrofysik, som har vist sig meget relevant for dette projekt.

%% Table of Contents %%
\phantomsection													% Kunstigt afsnit, som hyperlinks kan 'holde fast i'
\pdfbookmark[0]{Indholdsfortegnelse}{indhold}					% Tildeler en klikbar bookmark til den endelige PDF
\tableofcontents*												% Indholdsfortegnelsen (kaldet ToC)

	\end{folderinput}

\mainmatter %% Normale sidetal %%
\chapter{Indledning}
{\color{red} Denne skal laves om! Kommer senere!} Måske er det "Introduktion: Trådløs energioverførsel, som skal være her?

I dagens samfund benytter man strøm til næsten alt, der bliver brugt strøm til at styre alle vores teknologiske apparater, strømmen bliver sendt ud til apparaterne via ledninger, hvorimod man kunne gøre dette igennem trådløs overførelse. Trådløs energi overførelse er noget rimeligt nyt, der ses i hverdagen, men teknologien stammer tilbage fra 1890´erne hvor Nikola Tesla havde allerede forsøgte sig med, at skabe og sende trådløs elektricitet, (Bellow,2016) og drømte om at sende energien igennem den øvre atmosfære. Nikola Tesla drømte om fri energi til hele verden ikke blot kun til en by, men der er stadig lang vej den dag i dag. Metoder der bliver brugt i dag har stadig utrolig mange problemstillinger før det vil være muligt og erstatte traditionelle kabler. Før det er muligt, at lade trådløs elektricitet overtage hverdagen, er der nogle krav til denne teknologi 

Der er mange forskellige måder og sende energien på, men et af kravene der er meget vigtige er, at disse metoder ikke gør skade på mennesket, ikke mindst med det er det vigtig at energi kan blive sendt over længere distancer uden for meget spild og ustabil forbindelse. I dag er der meget fokus på global opvarmning og ikke mindst grøn energi, hvor Danmark har 2020 og 2050 planerne derfor er det også vigtigt, at der er tænkt miljøbevist før teknologien ville kunne slå igennem og blive den mest anvendte.

Der er to forskellige tilgange til denne teknologi, WSP (Wireless Power Transfer). En af tilgangene benytter sig af højfrekvens bølger, som mikrobølger eller lasere, en af disse sendes gennem luften hen til en modtager der kan omdanne den modtaget stråle/mikrobølge energien (photonerne) til elektricitet igen. Hvis denne metode anvendes er det muligt, at sende energi over længere afstande uden nogen form for fysik tilkobling, dog er det hovedsaglige problem med denne teknologi, at hvis der kommer noget imellem afsenderen og modtagerne ophøre overførelsen, ikke desto mindre kan det være skadeligt for mennesker, at udsætte dem for strålerne. Den anden tilgangsmåde forholder sig lidt anderledes her anvendes der elektromagnetisme, denne tilgangsmåde benytter sig af de elektroner der løber gennem en ledning, når elektroner løber gennem en ledning skabes der et magnetisk felt omkring (Amperes lov).  Når et magnetisk felt får indflydelse på en ledning skubber det magnetiske felt  til elektronerne, som skaber elektricitet (Faraday´s lov). I sådan et system har i forhold til den anden tilgang en begrænset rækkevidde, men den er alt mere sikker at anvende for mennesker. Denne teknologi bliver i dag brugt til flere forskellige produkter, eksempelvis bliver det brugt til og lade elektriske tandbørster op med og ikke mindst til pacemakers. Dette er denne type teknologi projektet vil fokuser på. 
 %% Står hernede da den rodede rundt i kapitlerne %%

	\begin{folderinput}{Problemanalyse}
\chapter{Problemanalyse}
\section{Introduktion: Trådløs energioverførsel}

Noget der kommer lige efter indledning:

Hvor ser man wpt i dag?

Trådløs energioverførsel eller trådløs opladning er i dag implementeret i forskellige produkter bl.a. mellem en elektrisk tandbørste og dens opladerstation. Til selve energioverførslen benyttes induktiv kobling oftest, da det effektivt er muligt at overføre energien over en kort afstand. Induktiv kobling er en form af elektromagnetisme. Andre typer indenfor elektromagnetisme som mikrobølger kan også benyttes til trådløs energioverførsel, men da mikrobølger har en skadelig effekt på levende organismer, og induktiv kobling allerede bliver benyttet til formålet energioverførsel, så ser vi derved nærmere på denne type af elektromagnetisme.

**(MIT-artikel)** Flere fordele for induktiv kobling...

Hvorfor er det relevant at kigge på wpt i dag?

Man kan altid spørge om, hvor vi ser teknologien i fremtiden, men et andet vigtigt perspektiv er, hvor relevant trådløs energioverførsel er i dag. Vi er et progressivt folkefærd, og vi vil altid gerne anerkendes for vores projekter. Vi er nået så langt, som vi kan nå med ledninger, så det næste skridt må være at fjerne ledningerne. Med den store ændring i regeringernes klimaaftaler, skal vi også tænke på transport. Vi bliver nød til at finde en måde at få vores elbiler til at køre længere, en måde at kunne gøre det kan være ved at forbedre batterierne eller lade bilen lade op, imens den kører. Her kan man også snakke om busser, som når de holder og samler passagerer op, kan nå at lade lidt op inden den kører videre. Ved at busserne hele tiden kører rundt, er de en af de store syndere for CO2, og det kunne være en løsning på at skære ned på CO2-udledningen, vi har i dag.

Hvor ser vi at vi kan bruge det henne?

Teknologien giver base for en række nye muligheder for brug af elektriske produkter; ikke kun som gadgets, men også indenfor transportmidler og større maskineri på fabrikker. Fremadrettet ville der kunne skabes mulighed for ét samlet energisystem, der kan registrere, hvis ens elektriske produkter mangler strøm for at være fuldt opladt. Herfra kan der blive overført strøm fra en transmitter til ens elektriske produkter, uden man skal tilslutte dem en ledning i en stikkontakt. Dette kunne bl.a. implementeres i ens hjem, hvor transmittere ville kunne oplade din mobil ligegyldigt, hvor i huset man befinder sig, samtidig med at det ville kunne drive køleskabet i køkkenet og tv'et i stuen.

Set i et andet perspektiv, ville teknologien også gøre elektriske transportmidler som biler og busser mere attraktivt. Hvis transmittere bliver implementeret i vejnettet, ville man kunne oplade sin bil, mens man kører. Det vil også kunne spille godt sammen med Elon Musks planer for at have Tesla til at lave busser som kunnes skiftes ud.

En anden ting kunne være, at man ikke længere skulle trække store mængder kabler gennem fabrikshaller eller bare den almindelige husstand. Dette ville skabe en mere mobiliseret produktion, hvor fabrikkerne ikke skal tage højde for, hvordan maskinerne kan placeres, så der er strøm til hele produktionen uden at ledninger og kabler løber på tværs af fabrikshallen.

 - Hvorfor har vi valgt at undersøge mobilopladning, i forbindelse med hvor vi forventer at se teknologien senere.
 
Vi har valgt at arbejde med trådløs mobilopladning, da vi ville bruge det som et "springbræt" til måske at arbejde videre med det. En anden grund til at det også er interessant er at det er i et tidligt stadie så vi kan nå at følge med på foreste række og bedre følge udviklingen. Det giver også beder mening at starte ved den teknologi, vi har i dag, frem for at vi kaster os ud på ny og ukendt grund uden baggrundsviden for, hvordan trådløs energioverførsel hænger sammen.

\newpage
\input{historie_old}
\input{basis}
\subsection{WPT metoder}
De tre relevante metoder til trådløs energi, indebære induktiv magnetisk-, resonans magnetisk kobling og mikrobølger. Induktiv og resonans virker ved near-fields og mikrobølger virker ved far-ifelds.

\begin{figure}[H]
\centering
\includegraphics[scale=0.5]{Vildledning/Schematics/induktiv_resonans}
\caption{model af WPT setup}
\label{model af WPT setup}
\end{figure}

Induktiv kobling. 
WPT ved induktiv kobling fungere ved induktion over et magnetfelt mellem to spoler, hvilke genereres en strøm og spænding som ses på figure 2.4a. Magnetisk induktiv kobling sker, når den primærspole som fungere som energi senderen genererer et vekselene magnetfelt på tværs af den sekundærspole, der er energien modtageren. Dette sker inden for et området generelt mindre end bølgelængden. Denne process skaber en near-field kraft som induktiveres over den sekundære spole til en strøm/spænding, hvilke kan udnyttes af et trådløs apparatur, eksempelvis en mobil. 
Fordelen ved induktiv kobling inkludere følgende, teknologien er simple og nem at implementer ved høj effektivitet og det er sikkert for mennesker at være i nærheden. Den er dog begrænset af afstand fra transmitter og receiver, idet at den har en effektiv energioverførelse mellem nogle få millimeter til et par centimeter. Ulemper ved induktiv kobling indebære den relative korte ladeafstand, varmen der udvikles i spolerne og spolerne skal ligge meget tæt og så lige overfor hinanden som muligt for at opnå den optimale effekt.
Teknologien bliver i dag set i mobile enheder (mobiltelefoner og tablets), tandbørster, RFID-tags, induktionskomfur og betalingskort.

Resonans kobling. 
Magnetisk resonans kobling er baseret på kortvarige bølgekobling, hvilke genere og overføre en elektrisk strøm mellem to resonans spoler i et oscillerende eller varierende magnetfelt, som set på figure 2.4b. Da de to spoler er køre på samme resonans frekvens, kan der opnå en høj effektiv elektrisk energioverførelse, med meget lidt tab til  eksterne faktorer. Denne indskab gøre energioverførelsen næsten upåvirket af omgivelserne, hvilke også gør det muligt at lade selvom der er noget i mellem transmitter og receiver. 
Den klare fordel ved resonans kobling er den meget større effektive lade distance, som kan opnå en effekt på 90 procent optil 1 meter mellem transmitter og receiver, og 40 procent ud til to meter. Derudover kan en enkel transmitter lader flere receiver på samme tid så længe de operer på sammen resonans frekvens.
En ulemperne ved resonans kobling er at hver receiver kræver en dedikeret kapacitets spole, hvilke gøre det svært at gøre receiveren lille nok til mobile enheder, og det gøre det generelt mere kompliceret at implementer. 
Fysikeren Michael Faraday beskrev, hvordan et skiftende magnetisk flux er i stand til at danne en elektrisk strøm. Den magnetiske flux angiver størrelsen på magnetfeltet i forhold til det givne areal, magnetfeltet påvirker. Den magnetiske flux afhænger af forskellige parametre, som spiller ind på at danne den elektriske strøm.

Den magnetiske flux er styret af arealet, den påvirker. Hvis størrelsen på det givne areal ændres, vil der også ske en ændring i den magnetiske flux. Hvis det gældende areal mindskes, vil den elektriske flux mindskes. Modsat vil en forøgelse af arealet medvirke, at den magnetiske flux øges.

%Evt. billede af forskellige størrelser areal

Størrelsen på arealet, der bliver påvirket, er ikke kun afgjort af arealets dimentionen, men også vinklen for, hvordan den magnetiske feltlinjer står ind på det givne areal. Hvis feltlinjerne står vinkelret på det gældende areal, vil flest mulige feltlinjer rammen arealet. Hvis arealet står vinklet under 90 grader, vil nogle af feltlinjerne ikke løbe igennem og påvirke arealet, og den magnetiske flux mindskes. Er arealet parallelt med de magnetiske feltlinjer, så vil så få magnetiske feltlinjer som muligt påvirke arealet. Når vinklen ændres for de magnetiske feltlinjer og arealet, så vil den magnetiske flux også ændres.

%Billede af magnetiske feltlinjer og areal

Størrelsen for de magnetiske feltlinjer er påvirket af strømstyrken for systemet. Ved en ændring af strømstyrken, vil de magnetiske feltlinjer ændre størrelse, hvorved den magnetiske flux vil varriere.
\section{Spoler}

En spole er en noget der bliver brugt til mange forskellige ting, men det spolen bruges til er, at der sendes strøm igennem spolen, når strømen er blevet sendt igennem opstår der et magnetfelt om spolen. Magnetfeltet der bliver dannet omkring spolen, vil ligne det fra en magnet, men skal være af samme form så spolen. Feltet bliver meget kraftigere hvis der er jern inde i spolen. 

Hvis spolen kortsluttes mens der er strøm der løber igennem, vil spolen på bedst muligvis forsøge at opretholde strømmen, men da strømmen ikke kan løbe i en kreds kan det ikke lade sig gøre for spolen og opretholde strømmen.  Strømmen bruges i stedet i feltet, det vil sige at spændingen over spolens ender stiger voldsomt og der opstår ofte en gnist. Det er det samme der sker i tænd spolen i bilen eller i et køkken apparat eller lignende.

\begin{figure}[htbp]
	\centering
	\includegraphics[width=0.5
	\textwidth]{Vildledning/Schematics/magnetfelt_omkring_en_spole.png}
	\caption{Magnetfelt omkring en spole.\cite{spoler}}
	\label{spole1}
\end{figure}

En magnet der er permanent er også en metalgering, som der ofte er jern i, magneten for derved den egenskab, at den bliver magnetisk.  Bliver en permanent magnet skubbet over til noget der er kortsluttede, det kan være en spole sker der det, at den permanente magnet vil fremkalde strøm i spolen, når der er fremkaldt en strøm vil den permanente magnet forsøge, at holde feltet ude. Det betyder at spolen forsøger, at lave poler, som vil vende modsat af den permanente magnet. Men strømen vil i løbet af meget kort tid forsvinde, hvis magneten ligger stille dette skyldes, at energien/strømen vil blive omdannet til varme inde i spolens modstand. Hvis magneten derimod bliver fjernet sker der omvendte, altså vil den påbegynde i det modsat retning, altså starte fra det felt der var inden magneten blev tilføjet. Ud fra dette kan det konkluderes, at der kun er strøm i spolen når der enten sættes et batteri til eller mag-netfeltet ændres. Hvis feltet forbliver konstant vil strømen forsvinde hurtigt og gå i nul dette skyldes også den ohmske modstand. 

En spole har ud fra dette altså ingen magnetfelt, medmindre den bliver påvirket af strøm eller man påvirker den vedhjælp af magnetisme. 

Der findes en række stoffer ved lave temperaturer disse stoffer er superledende, som også betyder, at de ingen elektrisk modstand har, når de ingen elektrisk modstand har ligger modstanden på $0 \Omega$. Der er tale om legeringer af stoffer der er sjældne, men er almindelig kendte i blandt andet bly og kviksølv. Laves der en blyring, bly ringen bliver sat ved stuetemperatur hvor der bliver sat en permanent magnet igen-nem, hvis nu det hele bliver kølet ned til det absolutte nulpunkt som ca. er $-273^\circ C$, når dette er gjort fjernes den permanente magnet, når den permanente magnet er blevet fjernet vil der ske det, som normalt sker ved induktion der vil nemlig opstå en strøm i blyringen, den strøm der opstår vil sørger for og gen-skabe det magnetfelt, som den permanente magnet havde. Der er ingen modsat hvilket også vil gøre, at strømmen vil forsætte så der haves et konstant magnetfelt, som er magen til det oprindelige magnetfelt. Magnetfeltet bliver målt i en enhed kaldet Tesla, som er opkaldt efter Nikola Tesla 1856-1943. \cite{spoler}


\input{limits} %% QI er inkluderet her %%
\section{Problemformulering}

\begin{itemize}

\item Hvordan kan resonant frekvens benyttes til at forbedre afstanden hvor ved den trådløse opladning i en mobiloplader er effektiv.

\end{itemize}


	\end{folderinput}
	
\begin{folderinput} {Teori}
\chapter{Teori}
\section{Reaktans til Impedans}
Reaktans:

Ved et LCR-kredsløb, så er der ikke kun tale om modstanden fra resistoren, men hvilken modstand kapacitoren og induktoren har. Den samlede modstand for kredsløbet kan beskrives igennem kompleks impedans, som er den samlede modstand for resistoren, kapacitoren og induktoren på kompleks form. Impedansen er essentiel til beskrivelse af den faseforskydelse, der kan opstå mellem spændingen og strømstyrken i LCR-kredsløbet.

For at forstå modstanden for en kapacitor og induktor, så skal begrebet reaktans bringes på banen. Ved elektriske felter er der tale om kapacitiv reaktans, som er gældende for kapacitorer, mens der for magnetiske felter er tale om induktiv reaktans, som er gældende for induktorer. Reaktansen for henholdsvis kapacitoren og induktoren er angivet som deres modstand og optræder efter sammenhængen mellem strømstyrken og spændingen over kapacitoren og induktoren. Da de to komponenter ikke optræder ens i kredsløbet, er beregningerne for reaktansen forskellig.
\begin{equation}
Kapacitiv reaktans: X_C = - \frac{1}{j \omega C}
\end{equation}
\begin{equation}
Induktiv reaktans: X_L = j \omega L
\end{equation}

Ved ovenstående formler indgår der to konstanter C og L. C er en konstant for kapacitoren angivet som kapacitansen, mens L er en konstant for induktoren angivet som induktansen. Omega beskrives som $\omega = 2 \pi f$, hvor $2 \pi$ svarer til én svingning, mens f er frekvensen. j angiver, at reaktansen er beskrevet som et kompleks tal.

Komplekse tal bliver benyttet for bedre at kunne knytte spænding og strømstyrke sammen, så det kan afbilledes i forbindelse med den fastforskydelse, der kan forekomme. Dette forhold mellem spænding og strømstyrke kan udledes gennem omskrivning af sinus og cosinus, men for nemhedens skyld benyttes komplekse tal for at simplificere formlerne. Formlerne for reaktansen kan derved skrives om til:

\begin{equation}
Kapacitiv reaktans: X_C = - \frac{1}{j 2 \pi f C}
\end{equation}
\begin{equation}
Induktiv reaktans: X_L = j 2 \pi f L
\end{equation}
%Kilde: Reactance of capacitors and inductors

En perfekt kapasitor uden modstand ville forskyde bølgefunktionen for strømstyrken med $\frac{\pi}{2}$ frem for bølgefunktionen for spænding., hvilket svarer til en kvart svingning (1/4 frekvens. Modsat ville den perfekte induktor uden modstand sænke bølgefunktionen for strømstyrken med $\frac{\pi}{2}$ i forhold til bølgefunktionen for spænding.

Billeder af forskudte bølger (kapacitiv og induktiv reaktans)
\begin{figure}[H]
\centering
\includegraphics[scale=1]{Vildledning/Schematics/Kapacitiv_reaktans}
\caption{Kapacitiv reaktans}
\label{kreaktans}
\end{figure}
%Kilde: Capacitance in AC Circuit and Capacitive Reactance

\begin{figure}[H]
\centering
\includegraphics[scale=0.8]{Vildledning/Schematics/Induktiv_reaktans}
\caption{Induktiv reaktans}
\label{ireaktans}
\end{figure}
%Kilde: Inductive Reactance - Reactance of an Inductor

Da den komplekse impedans var LCR-kredsløbets samlede modstand, så kan det beregnes ved:
\begin{equation}
Z = R + j 2 \pi f L - \frac{1}{j 2 \pi f C}
\end{equation}
%Kilde: Complex Impedance
\newpage
\section{Magnetfelter og resonant frekvens}

Magnetisk induktiv kobling er i stand til at skabe en effektiv energioverførsel, hvor ca. 90 procent af den udsendte effekt bliver opfanget af modtageren. Dette kan samtidig gøres ved et lavt frekvensspektrum på 20 til 40kHz med en rækkevidde på op til 10cm. Magnetisk indukttiv kobling har en lille påvirkning på omkringliggende genstande, hvilket mindsker energispildet ved overførslen, men der er andre faktorer, der kan spille ind på, at effektiviteten for energioverførslen reduceres.

Placeringen for spolerne er ikke tilfældig, og det kan have en stor betydning for effektiviteten af energioverførslen, hvis spolerne ikke har direkte forbindelse. Det påvirker ikke magnetfelterne så meget, hvis ikkemagnetisk materiale er placeret mellem transistoren og modtageren. Magnetfelterne passerer gennem det ikkemagnetiske materiale, men bliver kun sænket af, at feltlinjerne magnetiserer partiklerne omkring. Til gengæld vil en forskydning af spolerne, eller at spolerne står vinklet på hinanden, reducere den optagede energi fra modtageren, da færre af de magnetiske feltlijner når modtageren. (Nicolai)

\begin{figure}[H]
	\centering
	\begin{minipage}[b]{0.48\textwidth}
	\centering
	\includegraphics[width=0.5\textwidth]{Vildledning/Schematics/forskudte_spoler} % Venstre billede
	\end{minipage}
	\hfill
	\begin{minipage}[b]{0.48\textwidth}
	\centering
	\includegraphics[width=0.5\textwidth]{Vildledning/Schematics/vinklet_spoler} % Højre billede
	\end{minipage}
	\\ % Figurtekster og labels
	\begin{minipage}[t]{0.48\textwidth}
	\caption{Forskudte spoler} % Venstre figurtekst og label
	\label{fspoler}
	\end{minipage}
	\hfill
	\begin{minipage}[t]{0.48\textwidth}
	\caption{Vinklet spoler} % Højre figurtekst og label
	\label{vspoler}
	\end{minipage}
\end{figure}

Resonant frekvens:

Når der løber en strøm gennem et elektrisk kredsløb med vekselstrøm, så er der tilsvarende en frekvens for, hvor hurtigt strømmen skifter retning i systemet. Ved induktiv kobling er frekvensen med til at bestemme outputtet for det elektriske kredsløb. Den specifikke frekvens for et kredsløb, der giver det største output, kaldes for den resonante frekvens. For at energioverførslen mellem transistoren og modtageren skal være mest effektiv, så skal begge kredsløb have en ens resonant frekvens. På denne måde bliver effekten udsendt med størst mulig output, og modtageren kan opfange effekten på bedst mulig vis.
\section{Faraday's lov}

Faraday's lov beskriver induktionen af elektricitet, ved hjælp af magnetisme. Herved omhandler det den magnetiske flux, i stedet for den elektriske flux, som bliver brugt ved Gauss's lov. Formlen for magnetisk flux er ens med formlen for den elektriske flux, dog hvor det elektriske felt er byttet ud med det magnetiske felt: $\Phi_B = \int \vec{B} \cdot \vec{dA}$

En induceret strøm opstår ikke fra den magnetiske flux alene, men ved en ændring i den magnetiske flux. Dette betyder, at der bliver induceret spænding, dvs. hvis der sker en ændring af magnetfeltets styrke, den påvirkede overflades størrelse eller vinklen for, hvordan det magnetiske felt går gennem den pågældende overflade.

Faraday benytter den magnetiske flux til, at beskrive den inducerede spænding ved \cite{fysikbog}:

\begin{equation} \label{faraday}
\centerline{$\varepsilon = -1 \cdot \frac{d \Phi_B}{dt}$}
\end{equation}

Ændringen af den magnetiske flux forekommer modsat af den inducerede spænding, så derfor ganges en faktor $-1$ på det differentierede udtryk af den magnetiske flux. Den magnetiske flux kan også beskrives som: $\vec{B} \cdot \vec{A}$ eller $B \cdot A \cdot cos(\theta)$.
\newpage
\end{folderinput}

	\begin{folderinput}{Lab}
\chapter{Forsøg}
\subsection{Antagelser}

Forsøg med LC-kreds:

Ved måling over kapacitoren, når der sker en stigning i frekvens, så vil der ske en stigning i spændingsfaldet over kapacitoren. Denne stigning vil kun øge en smule, til der nærmes den resonante frekvens for LC-kredsløbet. Spændingsfaldet over kapacitoren vil begynde at stige kraftigt nær den resonante frekvens, hvor den vil nå sit toppounkt. Herefter vil spændingen igen falde, hvis frekvensen overstiger den resonante frekvens. Hvis den resonante frekvens ikke opnåes, så vil der ske en lille stigning i spændingsfaldet over kapacitoren.

Forsøg med LCR-kreds (Resistor):
Resistoren ændrer ikke værdi gennem forsøget, men har en fast resistens i kredsløbet, hvorved spændingsfaldet burde holdes stabilt. Derved sker der ikke nogen ændring i spændingsfaldet over spolen, når frekvensen øges, og der vil forekomme en liniær rekretion for spændingsfaldet i forhold til frekvens, hvor spændingsfaldet forbliver uændret.

Forsøg med LCR-kreds (Induktor):
Frekvensen er antal svingninger igennem systemet, og jo flere svingninger des større magnetfelt vil blive dannet ved induktoren. Spændingsfaldet over induktoren vil derved stige i takt med, at frekvensen stiger. Dette vil forekomme som en liniær rekretion, hvor frekvensen og spændingsfaldet stiger i takt med hinanden.
\section{Forsøgsbeskrivelse}
Hej med dig
\section{Forsøg 2 - IKEA-oplader} \label{sec:forsg2}

I dette forsøg undersøges sammenhængen mellem afstanden af oplader (senderen) og telefonen (modtageren) med henblik på tabet af energi når der sker en trådløs opladning med en QI certificeret oplader. Specifikt når afstanden $(L)$ mellem sender og modtager vokser. Altså effektiviteten $\eta$ undersøges ved forskellige afstande mellem sender og modtager.

\subsection{Forsøgsbeskrivelse} 
\subsubsection{Komponenter}

\begin{table}[htbp] %% Komponenter tabel %%
\begin{tabular}{l|l|c|l}
        & Strømforsyning               & Oplader             & Cover               \\ \hline
Model:  & KMW-190-330-GS               & MORIK, E1404        & VITAHULT, 22974     \\
Input:  & 220-240V$\sim$50/60Hz, 0.18A & 19V$\sim$1.74A, 33W & 5 W       \\	
Output: & 19V$\sim$1.74A, 10W          & 5W                  & 5V, 1A             
\end{tabular}
\caption{Alt er aflæst direkte fra komponenternes etiketter.}
\label{table:sender}
\end{table}

%%%%%%%%%%%%%

\begin{table}[htbp]
\begin{tabular}{l|lcl}
         & Iphone 4 &          &        \\ \hline
Batteri: & Li-Po     & 1420 mAh & 5,3 Wh
\end{tabular}
\caption{Aflæst fra \cite{batteri}}
\label{table:batteri}
\end{table}

Strømkilden til strømforsyningen er et standart ikke jordet, tobenet, europæisk væg stikkontakt, $230$ volt $50 \, Hz$.

Som modtager er der benyttet en \textbf{Iphone 4}, \textit{Model A1332 EMC 380A med FCC ID: BCG-E2380A, IC: 579C-E2380A.}\footnote{Aflæst fra bagsiden af enheden.}
Denne enhed er udstyret med cover, model og egenskaber kan ses i tabel \ref{table:sender}. Her er det dog vigtigt at notere sig at telefonen er brugt, og er en ældre model (købt november 2011). Altså er batteriet ikke i fabriksny tilstand. Det er usikkert, om dette har haft en relevant effekt på det udførte forsøg. Da dette kan have en effekt på, hvordan batteriet lader. I forsøget er der antaget at opladningen sker lineært, mere om dette i følgende sektioner.

Ydermere er der benyttet papir, til at lægge imellem mobilen med coveret og opladeren for, at øge afstanden mellem dem, så det magnetiske felt forstyrres mindst muligt, og afstanden er holdt fast. 

\begin{figure}
\includegraphics[width=1\textwidth]{Vildledning/Schematics/forsg2_opstilling1}
\caption{Opstilling med Iphone 4 og IKEA oplader L = 0,25 cm}
\label{figure:opstilling}
\end{figure}

\subsubsection{Fremgangsmåde}

Udførelsen af forsøget er meget simpel, og består af to dele der gentages tre gange. Første del er at bestemme, hvor højt modtager telefonen skal være løftet fra opladeren. Altså længden L bestemmes i cm, hvor positiv retning er væk fra opladeren. Anden del er at aflæse telefonens batteriprocent hvert tiende minut. Dette gentages med tre forskellige værdier for L, altså tre forskellige afstande.

Alle tre gange sikres der, at koblingen ikke er ustabil. Hjælp til at opnå dette er en diode, i opladeren, der lyser konstant hvis koblingen er stabil. For at opnå denne stabile kobling placeres telefonen midt ovenpå opladeren, så spolerne i henholdsvis sender og modtager ligger direkte ovenpå hinanden. Hertil er der tegnet en rød streg på både oplader og cover, det er muligt at afgøre at coveret ligger samme sted under hvert forsøg.

Første L værdi er valgt til $0 cm$, da dette teoretisk set burde være den optimale kobling mellem sender og modtager i et system designet til trådløs energioverførsel. Det ses også i den første graf, hvor $L = 0$ at denne opnår den højeste ladningsprocent. Det skal dog noteres, at spolerne i sender og modtager ikke lægger præcist direkte ovenpå hinanden, da der er et lag plastik og et gummikryds imellem. Disse antages ikke at have nogen effekt på opladningen. Længden L starter fra den hvide plastikskive med gummikrydset på opladeren, altså direkte ovenpå. Den præcise afstand mellem plastic og spole i senderen er ukendt. Det samme gælder i coverets ydre plastiklag, hvor dens tykkelse og afstand mellem den og modtager spolen også er ukendt.

Tiden er målt hvert tiende minut (bilag \ref{bilag:forsg2}), og målt med en afvigelse på $\pm 5$ sekunder, hvilket burde være hurtigt nok til ikke at se en ændring i ladningsprocenten.

Til slut undersøges værdien af L, hvor der ikke er muligt at danne en kobling, altså der hvor dioden stopper med at lyse, og der ikke sker en opladning af batteriet.

\subsubsection{Resultater}
Det ses at op til omkring L = 1 cm, at det ikke er muligt at danne en forbindelse mellem opladeren og coveret med telefonen i.

Rå, ikke redigeret, data kan ses i bilag \ref{bilag:forsg2}. Disse data er benyttet til at opstille de følgende tre diagrammer:

\begin{figure}[H]
\centering
\includegraphics[width=1\textwidth]{Setup/forsg2_graf1}
\caption{}
\label{figure:graf1}
\end{figure}

\begin{figure}[H]
\centering
\includegraphics[width=1\textwidth]{Setup/forsg2_graf22}
\caption{}
\label{figure:graf2}
\end{figure}

\begin{figure}[H]
\centering
\includegraphics[width=1\textwidth]{Setup/forsg2_graf3}
\caption{}
\label{figure:graf3}
\end{figure}

I første forsøg, hvor $L = 0$ ses det at opladning af batteriet nåede op på 82 procent ladning af maksimal ladning på batteriet $(100\%)$, indenfor den målte tidsperiode på 60 minutter. I de to næste forsøg ender opladningsprocenten efter 60 minutter på henholdsvis $81\%$ og $57\%$ ved $L = 0.25$ og $L = 0.5$ aflæst fra datasættet i bilag \ref{bilag:forsg2}, og ses tydeligt på figur \ref{figure:graf1}, figur \ref{figure:graf2} og figur \ref{figure:graf3}.

Der er lavet tre lineære regressioner per diagram, en af de tre første datapunkter, en for de tre sidste og den over dem alle samlet. Her ses det at opladningen ikke sker helt lineært, altså at opladningen sker ved samme hastighed gennem hele forløbet. Ved sammenligning af hældningskoefficienten af de tre første punkter og tre sidste punkter ses det, at opladningen er mere effektiv i starten af opladningen, end den er til slut f.eks. i figur \ref{figure:graf1} hvor $0,012\, \frac{E_\%}{t} > 0,009\, \frac{E_\%}{t}$.

Den ikke lineære opladning kan skyldes af et væld af ting, hvor det er antaget at det ikke er skyldt af den trådløse opladning, men snarere faktorer som: et ældre batteri, smartchargin teknologi\footnote{Det er ikke undersøgt om den målte telefon har disse teknologier eller hvordan de fungerer, på højere end basis niveau.} eller andet software der har en effekt på batteriet el. andet. 

\subsection{Databehandling}
I denne sektion undersøges det, ud fra det begrænsede datasæt, hvor stor effektivitet der mistes ved at forlænge afstanden mellem oplader og telefon med cover. Dette undersøges ved at beregne forskellige nyttevirkninger for den varierende værdi af L. 

Det antages at når batteriets batteri-procent kan aflæses til $100 \%$ på telefonens display at batteriet indeholder $O_{maks}=1432\, mAh \approx 5,3 Wh$. Dette benyttes så til at udregne nyttevirkningen $\eta$ med watt time værdier.

Det aflæses i skemaet \ref{table:sender} at opladeren har en effekt på $5 W$, og coveret har en en effekt på $(5 V \cdot 1 A = 5 W)$. Hvis der antages, at der bliver ladt op telefonens batteri med den fulde mulige effekt gennem hele forsøget (lineær opladning), så kan der bestemmes hvilken procent batteriet skulle kunne aflæses til efter 60 minutter:

\begin{equation}
O_{maks60}= \frac{5 W\cdot 1h}{5,3Wh} \cdot 100 \approx 94 \%
\label{eq:omaks}
\end{equation}

Dette er så den teoretiske maksimale opladningsprocent, batteriet kan ende på. Denne viser også, hvor meget der kan lades på batteriet på 60 minutter:
\begin{equation}
\Delta O_{maks} = 5 W \cdot 1 h = 5Wh
\end{equation}

Altså er den maksimale mængde watt timer den givne oplader kan påføre batteriet $5 Wh$. I alle forsøgene starter mobilen på 15 \% opladning af de 5,3 Wh så:
\begin{align*}
O_{start} = 5,3 \cdot 0,15 = 0,8 Wh
\end{align*}
 

\textbf{1. kør (L=0)}

I det første forsøg ender telefonen på 82 \% efter 60 minutter så:
\begin{align*}
& \Delta O_{1\%} = 82\%-15\% =  67\%  \\
& \Delta O_1 = 5,3 Wh \cdot 0,67 = 3,6 Wh
\end{align*}


Nyttevirkningen (i \%) kan nu beregnes vha. $\Delta O_1$ og $\Delta O_{maks}$:  
\begin{equation}
\eta_1 = \frac{\Delta O_1}{\Delta O_{maks}} \cdot 100 = \frac{3,6 Wh}{5 Wh} \cdot 100 \approx 86 \%
\label{eq:nyt1}
\end{equation}

Dette gøres ved de to næste forsøg, stadig efter 60 minutters opladning.

\textbf{2. kør (L=0,25):}
\begin{align*}
& \Delta O_{2\%} = 71\%-15\% =  56\%  \\
& \Delta O_2 = 5,3 Wh \cdot 0,56 = 3 Wh
\end{align*}
Nyttevirkningen beregnes:
\begin{equation}
\eta_2 = \frac{\Delta O_2}{\Delta O_{maks}} \cdot 100 = \frac{3 Wh}{5 Wh} \cdot 100 = 60 \%
\label{eq:nyt2}
\end{equation}

\textbf{3. kør (L=0,5):}
\begin{align*}
& \Delta O_{3\%} = 57\%-15\% =  42\%  \\
& \Delta O_3 = 5,3 Wh \cdot 0,42 = 2,2 Wh
\end{align*}
Nyttevirkningen beregnes:
\begin{equation}
\eta_3 = \frac{\Delta O_3}{\Delta O_{maks}} \cdot 100 = \frac{2,2 Wh}{5 Wh} \cdot 100 \approx 44 \%
\label{eq:nyt3}
\end{equation}

De tre værdier for $\eta$ stilles op i et $(L,\eta)$ koordinatsystem ser det således ud som i figur \ref{figure:graf4}

\begin{figure}[H]
\includegraphics[width=1\textwidth]{Setup/forsg2_graf4}
\caption{}
\label{figure:graf4}
\end{figure}
I samme omgang laves der en lineær regression men kun af de tre målte punkter, dette aflæses til:
\begin{equation}
y=-0,8422x+0,8393 \Rightarrow \eta = -0,8422L+0,8393
\label{eq:regression}
\end{equation}
Det sidste punkt der er skæringen med x-aksen bestemmes ud fra regressionen \ref{eq:regression}:
\begin{align*}
& solve(\eta =-0.8422L+0.8393,L) \\
& \Leftrightarrow L = 1,003964 \approx 1 cm
\end{align*} 
Dette er så den maksimale værdi for L, altså den maksimale afstand mellem opladerens overflade og telefonen med cover. Dette virker som en brugbar projektion, da det er tæt på den maksimale L værdi, som der blev målt til under forsøget, der var lige under 1 cm. Dette vil så sige, at den bestemte lineære ligning til en vis grænse, kan benyttes til at vise en sammenhæng mellem afstanden og nyttevirkningen, og i sidste ende hvor meget energi man kan få ud af opladeren. I hvert fald for den givne IKEA oplader.

Altså i de tre forsøg formåede den trådløse oplader fra IKEA at overføre $3,6 Wh$, $3,0 Wh$ og $2,2 Wh$ som henholdsvis er en nyttevirkning på 86 \%, 60 \% og 44 \% for de tre forsøg. Altså det er tydeligt, at når afstanden L mellem sender og modtager vokser, så mindskes energioverførslen betydeligt. Ud fra de tre forsøg, og den dannede prognose ses det, at den benyttede IKEA (QI) opladers nyttevirkning ændrer sig ud fra ligning \ref{eq:regression}.
	\end{folderinput}
	
Wireless power transfer er ikke så veludviklet i dag, hvilket ikke gør det optimalt at implementere i produkter, før der er sket væsentlige forbedringer. Fra teknologiens standpunkt så kan det ikke betale sig at benytte trådløs opladning til afstande på mere end 10cm ved effetive transmittere og modtagere. Almindelige trådløse opladere i dag har brug for, at produktet skal ligge tæt op ad transmitteren for at få en stabil opladning. Ved en gennemgang af forsøget med den trådløse oplader fra IKEA, viser det sig, at der kan opstå usikkerheder ved en afstand på $0,5 cm$ og ved større afstande mellem transmitter og modtager, går signalet tabt i omgivelserne. Derudover ses der en sammenhæng mellem afstanden og effektiviteten på opladningen, hvor en forøgelse i afstanden mellem transmitter og modtager mindsker den overførte strøm til produktet.

Modsat bliver opladningen af ens mobiltelefon mere mobiliseret, da opladerkablet ikke skal trækkes fra en stikkontak og direkte til mobiltelefonen. Ved igen at se på opladeren fra IKEA, så er den bygget til at kunne indsættes i bordpladen på et spisebord eller lignende, så telefonen bare skal placeres på opladeren i bordpladen i stedet for at tilslutte den til et kabel. Ved at kunne overføre energien til mobiltelefoner trådløst, så sker der også en ændring på selve mobilen, da indgangen til opladerkablet bliver overføldigt. Dette kan bl.a. have den fordel, at telefonen lettere kan gøres vandtæt. Integrationen af trådløs energioverførsel til telefonen kan gøres simpelt ved indsættelse af et LCR-kredsløb, der skal have en tilhørende transmitter, som har samme resonante frekvens, hvilket gør overgangen fra de traditionelle kabler til trådløs energioverførsel enkel.

Der er stor forskel på, hvor effektiv energioverførslen er i forhold, hvilken frekvens transmitterens og modtagerens system kører på. For en optimal energioverførsel, så skal begge kredsløb have en høj effektivitet ved samme frekvens. Herfra er indstillingen af den operative frekvens afhængig af, hvor stor modstand, der er indkorporeret i systemet alt efter hvilket produkt, der skal oplades. Til mobilopladning som ikke behøver en stor strømstyrke overført, så kan der med fordel benyttes en stor modstand, hvorved frekvensområdet for en stabil overførsel bliver bredt. Dette gør, at der kan ske udsvingninger i frekvensen mellem transmitter og modtager, uden strømstyrken svinger betydeligt.

Hvor modstanden er med til at afgøre hvor stort en frekvensområde, der kan arbejdes indenfor, så er den rigtige frekvens betydende for, hvornår strømstyrken stiger og falder. LCR-kredsløb har en resonant frekvens, hvor det største output forekommer. For at benytte resonant kobling mellem transmitter og modtager, så skal begge kredsløb have en ens resonant frekvens. Resonant kobling vil sikre den mest optimale overførsel af energi, men hvis de to kredsløb ikke en ens resonant frekvens, kan der ske store ændringer i hvor meget strøm, der bliver overført.
\chapter{Konklusion}

Der kan konkluderes, at WPT's potentiale udvides bedst ved brug af en magnetisk induktiv kobling med resonant frekvens. Resonant kobling har en klar fordel i forhold til effekt over afstand, vist ved forsøg med IKEA oplader, hvor der er et stort tab, når afstanden vokser, og ved en afstand på kun $0,5 \, cm$ er nyttevirkningen faldet til $44\%$. Dette er selvfølgelig kun relevant for den given oplader, hvor andre oplader, der bruger samme princip, kan nå op på en effekt på $90\%$ ved en afstand på op til $10 \, cm$, i forhold til en oplader med resonant frekvens hvor der kan opnås en effekt $90\%$ ved op til en meter. Dog er det sværre at intrigere i små enheder, idet der skal intrigeres en kapasitor i modtageren og afsender. Resonant frekvens har også den fordel, at flere modtagere kan kobles op til samme afsender uden bemærkelsesværdig tab, så længe der er samme resonant frekvens. Ved brug af resonant frekvens kan der også opnås et højt energi output, men der kan kun være lille udsving i frekvens, før der sker for store tab i overførelsen. Det kan konkluderes, at resonant frekvens er fremtiden inden for WPT, når afstand og output er gældende.

\chapter{Perspektivering}

Trådløs energioverførsel er ikke et nyt koncept, men det er først i nyere tid, hvor det er begyndt at vinde indtog. Flere og flere får øjnene op for de forskellige anvendelser, der ligger i teknologien trådløs energioverførsel, og det er et stort forskningsemne. På trods af den stigende interesse er overførslen af energi ikke effektiv, og meget af energien går tabt ved overførsel over korte afstande. For videre at skabe rammer for interesse og fokuspunkter for udvikling af teknologien, kan det være godt at se på område, hvor det kunne være nyttigt at benytte teknologien.

Opladning af elektroniske gadgets og hjælpemidler:
 	
Trådløs energiopladning giver grundlag for en mobiliseret dagligdag, hvor elektriske produkter ikke længere er afhængige af at skulle tilsluttes en stikkontakt, når de er ved at løbe tør for strøm. På den kortere bane kan der fokuseres på opladning af mindre elektriske gadgets som f.eks. mobiltelefoner eller pacemakers. Når teknologien er udviklet nok til, at der kan overføres en svag strøm over lange distancer, så kunne der placeres rutere (lignende rutere for netværksforbindelse wifi) rundt omkring i bylandskabet, så ens elektriske apparater kunne oplades på vej til og fra arbejde. På mindre skala kunne det være, at busser, indkøbscentre eller almindelige husstande kunne udsende energi trådløst til de elektroniske produkter, der befinder sig indenfor rækkevidde. Herved bliver batteriers levetid mindre betydende, da de passivt kan blive opladt nær rutere eller lignende, så længe det elektriske apparat er inden for ruternes radius. Dette gør, at ledninger bliver overflødige for mindre produkter. Opladning af gadgets ved hjælp af trådløs energioverførsel kan ses som en luksus, men det kan også være med til at afhjælpe personer, der benytter elektrisk apparatur for egen sikkerheds skyld. Herunder kan der være tale om opladning af pacemakers eller måleinstrumenter til personer med sukkersyge, som er afhængige af, at produkterne er opladt og klar til brug.

Kommunikation mellem produkt og transmitter:
 	
Som udbygning af konceptet med at oplade mobiltelefoner og andre gadgets, så skal der også tages højde for, hvornår og hvor meget transmitteren skal udsende af energi til de enkelte produkter. Transmitteren kan i princippet fortsætte med at overføre energi til produkterne, men overførsel af unødvendig elektricitet vil ikke være økonomisk favorabelt. Derudover vil det overbelaste og skade batterierne, hvis det konstant tilføres strøm, hvis produktet ikke har indbygget en begrænsning for opladningen. En idé til at forhindre overførsel af energi, som ellers ville gå til spilde, kunne være, hvis produkter kombatible til trådløs opladning kunne kommunikere med transistoren. Herved kunne transistoren opfange data omkring apparaternes batteriniveau og kun udsende strøm mellem en bestemt batteriprocent.

Implementering i bygninger og fabrikshaller:
 	
Som WPT udvikler sig, bliver det nemmere at overføre elektricitet og større mængder af det. Opladning af mobiltelefoner er et springbræt for udviklingen og kan derefter tages til nye højder. På et tidspunkt ville det ikke være nødvendigt at trække kabler gennem hele bygningskonstruktioner, men kun hen til, hvor transmitterne for den trådløse energioverførsel skal placeres. Organisatorisk vil dette være banebryden bare at indføre i almindelige husstande. Hvis køleskabet, komfuret og TV'et blev holdt igang vha. trådløs energioverførsel, så skal det ikke placeres nær stikkontakter, men det kan placeres, hvor det ønskes, så længe det befinder sig indenfor transmitterens rækkevidde. Det gør det lettere at flytte rundt på elektriske apparater, og de kan placeres i forhold til mindre pladsspild, da der ikke skal trækkes stikdåser til opkoblingen. Ved husstanden har det måske ikke den helt store betydning for at udnytte pladsen, men for værksteder eller fabrikker, kan det påvirke meget for produktionen for, hvordan maskiner drevet af elektricitet er placeret i forhold til hinanden. Hvis en udsender er ophængt under loftet, så den kan drive de forskellige maskiner, så skal der ikke trækkes kabler på tværs af fabrikshallen, og det giver større frihed for placeringen af maskinerne. Dette kan have stor indvirkning på fabrikkens effektivitet, da maskinerne bedre kan opstilles i forhold til produktionens behov.

Opkobling til trådløs energioverførsel:
 	
Ved indførelse af WPT i dagligdagsbilledet stiller det et spørgsmål til, om alle elektriske apparater har forbindelse til hver transmitter, eller om produkterne skal kobles med den enkelte sender. Ved offentlige områder kan det være der benyttes åbne netværker, som ved områder med gratis wifi, men ved private områder og husstande kan det give komplikationer, hvis alle har adgang til opladning af f.eks. mobiltelefoner. Derved kunne det med fordel være godt at kunne sætte adgangskode eller anden form for sikkerhed på signalet fra transmitterne, så der kan kontrolleres, hvem der har adgang. Hvis der stadig skal betales for den strøm, der bliver overført fra transmitteren, så kan det gå ud over de enkelte boliger, hvis folk på gaden kan få forbindelse til deres netværk og forbruge strøm.
 
Elektriske transportmidler:
 	
Der forekommer flere og flere elektriske produkter på markedet, og et punkt som udvikler sig hurtigt i dag er elektriske køretøjer som elbiler. Dette er også et punkt, hvor trådløs energioverførsel kan benyttes. Elektriske biler skal kobles til opladerstationer for at blive genopladt, men WPT kan gøre det muligt at oplade bilerne, mens de befinder sig på vejene. Især elektriske busser er på tale i dag, hvor der bliver set på, om busserne kan oplades omkring stoppesteder eller andre pladser, hvor bussen holder stille. Et andet koncept er at integrerer transmitterne i selve vejnettet, hvor vejene er inddelt i sektioner, som kan tændes separat. Herfra registreres busserne, hvor de er, hvorefter sektionen, de befinder sig på, begynder at overføre energi til bussen. På denne måde bliver bussen opladt, mens den kører gennem byen. Denne benyttelse af trådløs energioverførsel kan gøre det mere attraktivt at investere i elektriske køretøjer i stedet for transportmidler, der kører på fossilt brændsel.
 	
Energiindustrien:
 	
Set i det store perspektiv, når WPT kan overføre store mængder energi over lange afstande, så kan det begynde at benyttes ved energisektoren. Her kan der f.eks. ses på havvindmølleparker. Det kræver et kæmpe arbejde for at anlægge havvindmølleparker bl.a. i forbindelse med opkoblingen til fastlandet. Der skal trækkes kilometervis af kabler på havbunden samt transformerstationer for at holde strømmen stabil. Derudover er det et logistisk problem, hvis der skal foretages reparationer på kablerne. Alt tal sammen er også set som en stor udgift og arbejdskraft, som kan mindskes betydeligt, når WPT bliver udviklet nok til at implementeres i stedet for. Vindmøllerne kunne forbindes et en eller få transmittere ved selve parken, som ved induktiv kobling kunne overføre strømmen til transformerstationer på fastlandet, som derfra ville kunne føre strømmen videre til forbrugerne.

Delkonklussion:

Trådløs energioverførsel er ikke så veludviklet i dag, men det er et stort forskningsemne, som hele tiden forandre sig. Senere vil der optræde et væld af muligheder for brugen af teknologien, som dækker over et bredt spektrum fra mobilopladning i ens egen husstand til overførsel af energi fra havvindmølleparker til fastlandet. Med hjælp fra WPT vil transportsektoren kunne ændre sig til, at elektriske køretøjer får en længere rækkevidde og kan blive mere favorabel i bymiljøet. Der optræder dog andre problemstillinger i forbindelse med, at teknologien forbedres, som skal tages højde for på længere sigt.

	
\bibliography{bib/bibfil} %% Litteraturlisten indsættes her %%


\appendix
\clearforchapter
\phantomsection
\pdfbookmark[0]{Appendiks}{appendiks}
\chapter{Forsøg 1} \label{bilag:forsg1}

P1 - Gruppe C-16a - 07-Nov-16

\begin{figure}[htbp]
\centering
\includegraphics[width=1\textwidth]{Vildledning/Schematics/Eks1_LCR.png}
\caption{Foreslåede eksperiment kredsløb. R-kreds - CR-kreds - LCR-kreds}
\label{fig:Eks1}
\end{figure}
\newpage

\section{Hvad vil vi opstille?}
\begin{itemize}
\item R-kreds
\item CR-kreds
\item LCR-kreds
\end{itemize}
Til alle opstillingerne vil der i starten benyttes et oscilloskop som kilde, sat til vekselspænding ved 5 V, 50Hz.

De tre opsætninger benyttes som reference punkter til videre eksperimentring.  
\section{R-kreds}
Til at starte med vil vi opstille, det måske aller simpleste kredsløb, ved bare at sende en vekselspænding over en resistor og måle spændingsforskellen og strømmen over det.

Herefter vha. $U=R\cdot I$ kan der så undersøges om den aflæste værdi af resistoren er den samme som den målte.
\section{CR-kreds}
Opstillingen her er ens med R-kredsen, uden ampere-meter, bare at der nu er sat en kapacitator på og volt-meteret er flyttet hen over kapacitatoren.

Herefter ses der på om der er sket en ændring i spændingsforskellen over kapacitatoren, sammenlignet med den mængde spænding der er tilført systemet.
\section{LCR-kreds}
Dette er den vigtigste kreds der undersøges, da der nu er lavet et loop ved at sætte en spole i kredsløbet. Volt-meteret der benyttes her skulle gerne være et der kan tegne grafer, specifikt $(U,t)$.

Kredsløbet er en forlængelse af de to andre. Volt-meteret er nu bare flyttet til hen over spolen, da dette er den komponent med størst relevans.

Her undersøges der den spændingskurve der kan tegnes over spolen. Denne kan herefter sammenlignes med kurven i givet fra en simulering i Plecs. Dette er selvfølgelig kun relevant hvis det er muligt at komme tæt på virkelige værdier i programmet.

\chapter{Forsøg 1} \label{bilag:forsg1}

\begin{figure}[H]
\centering
\includegraphics[width=1\textwidth]{Setup/Bilag_forsg1}
\caption{Rå data for forsøg 1}
\label{tabular:forsg1}
\end{figure}

\textbf{RC - kredsløb}

\begin{figure}[H]
	\centering
	\begin{minipage}[b]{0.48\textwidth}
	\centering
	\includegraphics[width=1\textwidth]{Vildledning/Schematics/kredslb/RC} % Venstre billede
	\end{minipage}
	\hfill
	\begin{minipage}[b]{0.48\textwidth}
	\centering
	\includegraphics[width=1\textwidth]{Setup/Graf1} % Højre billede
	\end{minipage}
	\\ % Figurtekster og labels
	\begin{minipage}[t]{0.48\textwidth}
	\caption{Opstilling af RC-kredsløb} % Venstre figurtekst og label
	\end{minipage}
	\hfill
	\begin{minipage}[t]{0.48\textwidth}
	\caption{Graf for RC-kredsløb} % Højre figurtekst og label
	\end{minipage}
\end{figure}
\newpage

\textbf{RL - Kredsløb}

\begin{figure}[H]
	\centering
	\begin{minipage}[b]{0.48\textwidth}
	\centering
	\includegraphics[width=1\textwidth]{Vildledning/Schematics/kredslb/RL} % Venstre billede
	\end{minipage}
	\hfill
	\begin{minipage}[b]{0.48\textwidth}
	\centering
	\includegraphics[width=1\textwidth]{Setup/Graf2} % Højre billede
	\end{minipage}
	\\ % Figurtekster og labels
	\begin{minipage}[t]{0.48\textwidth}
	\caption{Opstilling af RL-kredsløb} % Venstre figurtekst og label
	\end{minipage}
	\hfill
	\begin{minipage}[t]{0.48\textwidth}
	\caption{Graf for RL-kredsløb} % Højre figurtekst og label
	\end{minipage}
\end{figure}

\textbf{CL - Kredsløb Serie}

\begin{figure}[H]
	\centering
	\begin{minipage}[b]{0.48\textwidth}
	\centering
	\includegraphics[width=1\textwidth]{Vildledning/Schematics/kredslb/CL_serie} % Venstre billede
	\end{minipage}
	\hfill
	\begin{minipage}[b]{0.48\textwidth}
	\centering
	\includegraphics[width=1\textwidth]{Setup/Graf4} % Højre billede
	\end{minipage}
	\\ % Figurtekster og labels
	\begin{minipage}[t]{0.48\textwidth}
	\caption{Opstilling af CL-kredsløb} % Venstre figurtekst og label
	\end{minipage}
	\hfill
	\begin{minipage}[t]{0.48\textwidth}
	\caption{Graf for CL-kredsløb} % Højre figurtekst og label
	\end{minipage}
\end{figure}

\textbf{LC - Kredsløb Parallel}

\begin{figure}[H]
	\centering
	\begin{minipage}[b]{0.48\textwidth}
	\centering
	\includegraphics[width=1\textwidth]{Vildledning/Schematics/kredslb/LC_Parallel} % Venstre billede
	\end{minipage}
	\hfill
	\begin{minipage}[b]{0.48\textwidth}
	\centering
	\includegraphics[width=1\textwidth]{Setup/Graf5} % Højre billede
	\end{minipage}
	\\ % Figurtekster og labels
	\begin{minipage}[t]{0.48\textwidth}
	\caption{Opstilling af LC-kredsløb parallel} % Venstre figurtekst og label
	\end{minipage}
	\hfill
	\begin{minipage}[t]{0.48\textwidth}
	\caption{Graf for LC-kredsløb parallel} % Højre figurtekst og label
	\end{minipage}
\end{figure}
\chapter{Forsøg 2} \label{bilag:forsg2}

\begin{figure}[htbp]

\centering
\includegraphics[scale=1]{Setup/forsg_2_bilag}
\label{figure:forsg2}

\end{figure}

%\includepdf[pages={1}]{Setup/forsg_2_bilag.pdf} Includere en pdf af bilaget

%\chapter{PV-fremadrettet}
Nu da vi har lavet vores første forsøg, hvor vi har set på, hvordan spændingen skifter over en modstand, kapacitator og spole ved ændring på frekvensen, kan vi nu gå videre til vores forsøg om WPT. Vi fik chancen for at omlægge den teori, vi har læst om, ud til en forsøgsopstilling, hvor vi kunne aflæse forskellige resultater, der giver mulighed for videre arbejde. Det har ikke lokket os fra at skulle arbejde videre med projektet, snarere det modsatte. Vi har fået nyt blod på tanden og glæder os til de afsluttende forsøg. Forsøgene skal benyttes til databehandlingen, så vi kan sammenligne resultaterne, så vi har konkrete tal at arbejde ud fra, når vi skal videre med problemløsningen. Vores strategi for fremtiden er at vi vil skriftes til at lave forsøg og dobbelttjekke hinandens forsøgsopstillinger. Når den ene del af gruppen er i laboratorierne vil den anden del af gruppen ihærdigt skriver videre, laver beregninger, skematics osv. til rapporten. Det er vores målsætning at holde møde hver fredag, så vi kan holde trit med vores tidsplan, og at hvis vi kommer bag ud, er det muligt for de andre i gruppen at træde til og hjælpe et medlem. Dette er ingen individuel rapport, og vi skal alle sammen kunne stå inde for rapporten. Ergo er det vigtigt, at man hjælper hinanden, for hvis der er en som taber, taber hele gruppen. Det er lige meget hvis skyld, det er, for vi er alle i samme båd.

For at holde gejsten oppe i gruppen sørger vi for at holde hinanden i nakken, så der ikke er nogen, der får problemer og falder bagud i forhold til gruppen. Projektet er en samlet indsats, så det er vigtigt at få lavet opsamlinger i gruppen, så alle kan følge med, og alle har samme udgangspunkt for senere arbejde. Derudover diskuterer vi om, hvad der interesserer os ved projektet, og hvad vi gerne vil have ud af det færdige produkt, så det holder os fast på et endeligt mål. Hvis vi arbejder med det, vi finder interesse for, er det lettere at holde sig fokuseret på arbejdsopgaverne, og man får lyst til at gøre en god indsats (ikke kun for en selv, men også for gruppen som helhed).

Som gruppe bliver vi enige om, hvilke arbejdsområder vi skal undersøge og skrive om. Herefter inddeler vi os i mindre grupper for hver arbejdsopgave (2-3 mand pr. gruppe). Dette gør, at vi hurtigt kan få indsamlet brugbar viden og udført en stort stykke arbejde, men at vi samtidig ikke står alene med en opgave. Det at være i små grupper gør også, at man kan få flere input og idéer til, hvad man eventuelt kan bringe ind over opgaven, hvilket gør at projektet bliver mere nuanceret. For at holde styr på, hvor langt hver gruppe er, og hvad de hver især har skrevet, så holder vi jævnligt møder til at opsummere processen.

Da vi er igennem problemanalysen, så har vi fået dannet os en god grundviden om projektet, som vi videre kan benytte til forsøg, modeller og projektløsningen senere hen. Dette betyder også, at vi nu skal til at specificere os på enkelte dele af projektet, så vi får indsnævret vores undersøgelser.

\end{document}