\input{Setup/preamble_ida16-Mads.tex}

%% ^^Alt setup findes i preamble^^ %%
\raggedbottom


\begin{document}
\frontmatter %% romertegn som sidetal %%

%% folderinput er så man frit kan flytte filerne ned til de relevante mapper %%

	\begin{folderinput}{Setup}

\thispagestyle{empty}
\begin{flushright}
\vspace{3cm}

\phantom{hul}

\phantom{hul}

\phantom{hul}

\textsl{\Huge Trådløs energioverførsel med fokus på trådløs mobilopladning} \\ \vspace{1cm}

\rule{13cm}{3mm} \\ \vspace{1.5cm}
\vspace{1cm}

%\includegraphics[width=0.864\textwidth]{Vildledning/Schematics/Vinkelflux}

\vspace{2cm} 
\textsc{\Large P1 Projekt \\
Gruppe C2-16a \\
Energi\\
Aalborg Universitet\\
Den 19. december 2016\\}
\end{flushright}

\newpage

\cleardoublepage
% Dette er LaTeX-versionen af titelbladet for TNB studenterrapporter
% Filen kræver:
% Universitetets logo:  AAU-logo-stud-UK eller AAU-logo-stud-DK
% Synopsis: En fil ved navn synopsis.tex

% Udarbejdet af: Jesper Nørgaard (jesper@noergaard.eu) 10. april 2012
% Redigeret af Mathias S. Hansen (mat_elm13@hotmial.com) 9. november 2016

%\phantomsection
\thispagestyle{empty}
%\pdfbookmark[0]{Titelblad}{titelblad}

\begin{minipage}[t]{0.48\textwidth}
\vspace*{-25pt}			%\vspace*{-9pt}
\includegraphics[height=4cm]{billeder/AAU-logo-stud-DK-RGB}
\end{minipage}
\hfill
\begin{minipage}[t]{0.48\textwidth}
{\small 
\textbf{Første Studieår v/ }\\
\textbf{School of Engineering and Science (SES)}  \\
Energi \\
Strandvejen 12-14 \\
9000 Aalborg \\
http://www.tnb.aau.dk}
\end{minipage}

\vspace*{0.2cm}

\begin{minipage}[t]{0.48\textwidth}
\textbf{Titel:} \\[5pt]\bigskip\hspace{2ex}
Trådløs Mobilopladning

\textbf{Projekt:} \\[5pt]\bigskip\hspace{2ex}
P1-projekt

\textbf{Projektperiode:} \\[5pt]\bigskip\hspace{2ex}
September 2016 - December 2016

\textbf{Projektgruppe:} \\[5pt]\bigskip\hspace{2ex}
C2-16a	

\textbf{Deltagere:} \\[5pt]\hspace*{2ex}

\noindent\begin{tabular}{ll}
\makebox[2.5in]{\hrulefill} \\
Daniel Revsbech Pedersen \\
\end{tabular} \\[10pt]\hspace*{2ex}


\noindent\begin{tabular}{ll}
\makebox[2.5in]{\hrulefill} \\
Mads Lindstrøm Paulsen \\
\end{tabular} \\[10pt]\hspace*{2ex}

\noindent\begin{tabular}{ll}
\makebox[2.5in]{\hrulefill} \\
Mathias Stenberg Hansen \\
\end{tabular} \\[10pt]\hspace*{2ex}

\noindent\begin{tabular}{ll}
\makebox[2.5in]{\hrulefill} \\
Nicolai Nørgaard Munk \\
\end{tabular} \\[10pt]\hspace*{2ex}

\noindent\begin{tabular}{ll}
\makebox[2.5in]{\hrulefill} \\
Sisse Sorgenfri Jensen \\
\end{tabular} \\[10pt]\hspace*{2ex}

\noindent\begin{tabular}{ll}
\makebox[2.5in]{\hrulefill} \\
Torben Brund Jørgensen \\
\end{tabular} \\[10pt]\hspace*{2ex}

%Daniel Revsbech Pedersen \\[5pt]\\\hspace*{2ex}
%Julian Bo Larsen \\[5pt]\\\hspace*{2ex}
%Mads Lindstrøm Paulsen \\[5pt]\\\hspace*{2ex}
%Mathias Stenberg Hansen \\[5pt]\\\hspace*{2ex}
%Nicolai Nørgaard Munk \\[5pt]\\\hspace*{2ex}
%Sisse Sorgenfri Jensen \\[5pt]\\\bigskip\hspace{2ex}
%Torben Brund Jørgensen

\textbf{Vejledere:} \\[10pt]\hspace*{2ex}

\noindent\begin{tabular}{ll}
\makebox[2.5in]{\hrulefill} \\
Christian Uhrenfeldt \\
\end{tabular}
%Christian Uhrenfeldt \\\hspace{2ex}%\\\bigskip\hspace{2ex}

%\vspace*{1cm}

\end{minipage}
\hfill
\begin{minipage}[t]{0.483\textwidth}
Synopsis: \\[5pt]
\fbox{\parbox{7cm}{\bigskipThis project aims to inform the reader about the usage of wireless power transfer. The project discusses the use of WPT in the past, the present and a possible future. This project uses mathematical modelling to give the reader an insight in magnetic inductive coupling. This paper is founded on basic knowledge of electromagnetism in addition to wireless power transfer. This knowledge is used in experiments of a LCR-circuit, in order to prove the connection between frequency and voltage, and thus the peak current at resonant frequency. Furthermore, an experiment to show the loss of power at increasing distance between sender and reciever. This paper concludes that resonant magnetic coupling will provide the most efficient energy transfer and will open up for further research on the subject matter for a future of more effecient long range wireless power transfer.\bigskip}}

\vspace*{16.5cm}

\textbf{Oplagstal: ?} \\
\textbf{Sidetal: ?} \\
\textbf{Appendiks: ?} \\ 
\textbf{Afsluttet: DATO}

\end{minipage}

\vfill


%{\footnotesize\itshape Rapportens indhold er frit tilgængeligt, men offentliggørelse (med kildeangivelse) må kun ske efter aftale med forfatterne.}

% Rapportens indhold er frit tilgængeligt, men offentliggørelse (med kildeangivelse) må kun ske efter aftale med forfatterne.
% The content of the report is freely available, but publication (with source reference) may only take place in agreement with the authors.
\cleardoublepage

%% Table of Contents %%
\phantomsection													% Kunstigt afsnit, som hyperlinks kan 'holde fast i'
\pdfbookmark[0]{Indholdsfortegnelse}{indhold}					% Tildeler en klikbar bookmark til den endelige PDF
\tableofcontents*												% Indholdsfortegnelsen (kaldet ToC)

	\end{folderinput}

\mainmatter %% Normale sidetal %%
\chapter{Indledning}
{\color{red} Denne skal laves om! Kommer senere!} Måske er det "Introduktion: Trådløs energioverførsel, som skal være her?

I dagens samfund benytter man strøm til næsten alt, der bliver brugt strøm til at styre alle vores teknologiske apparater, strømmen bliver sendt ud til apparaterne via ledninger, hvorimod man kunne gøre dette igennem trådløs overførelse. Trådløs energi overførelse er noget rimeligt nyt, der ses i hverdagen, men teknologien stammer tilbage fra 1890´erne hvor Nikola Tesla havde allerede forsøgte sig med, at skabe og sende trådløs elektricitet, (Bellow,2016) og drømte om at sende energien igennem den øvre atmosfære. Nikola Tesla drømte om fri energi til hele verden ikke blot kun til en by, men der er stadig lang vej den dag i dag. Metoder der bliver brugt i dag har stadig utrolig mange problemstillinger før det vil være muligt og erstatte traditionelle kabler. Før det er muligt, at lade trådløs elektricitet overtage hverdagen, er der nogle krav til denne teknologi 

Der er mange forskellige måder og sende energien på, men et af kravene der er meget vigtige er, at disse metoder ikke gør skade på mennesket, ikke mindst med det er det vigtig at energi kan blive sendt over længere distancer uden for meget spild og ustabil forbindelse. I dag er der meget fokus på global opvarmning og ikke mindst grøn energi, hvor Danmark har 2020 og 2050 planerne derfor er det også vigtigt, at der er tænkt miljøbevist før teknologien ville kunne slå igennem og blive den mest anvendte.

Der er to forskellige tilgange til denne teknologi, WSP (Wireless Power Transfer). En af tilgangene benytter sig af højfrekvens bølger, som mikrobølger eller lasere, en af disse sendes gennem luften hen til en modtager der kan omdanne den modtaget stråle/mikrobølge energien (photonerne) til elektricitet igen. Hvis denne metode anvendes er det muligt, at sende energi over længere afstande uden nogen form for fysik tilkobling, dog er det hovedsaglige problem med denne teknologi, at hvis der kommer noget imellem afsenderen og modtagerne ophøre overførelsen, ikke desto mindre kan det være skadeligt for mennesker, at udsætte dem for strålerne. Den anden tilgangsmåde forholder sig lidt anderledes her anvendes der elektromagnetisme, denne tilgangsmåde benytter sig af de elektroner der løber gennem en ledning, når elektroner løber gennem en ledning skabes der et magnetisk felt omkring (Amperes lov).  Når et magnetisk felt får indflydelse på en ledning skubber det magnetiske felt  til elektronerne, som skaber elektricitet (Faraday´s lov). I sådan et system har i forhold til den anden tilgang en begrænset rækkevidde, men den er alt mere sikker at anvende for mennesker. Denne teknologi bliver i dag brugt til flere forskellige produkter, eksempelvis bliver det brugt til og lade elektriske tandbørster op med og ikke mindst til pacemakers. Dette er denne type teknologi projektet vil fokuser på. 
 %% Står hernede da den rodede rundt i kapitlerne %%

	\begin{folderinput}{Problemanalyse}
\chapter{Problemanalyse}
\section{Introduktion: Trådløs energioverførsel}

Noget der kommer lige efter indledning:

Hvor ser man wpt i dag?

Trådløs energioverførsel eller trådløs opladning er i dag implementeret i forskellige produkter bl.a. mellem en elektrisk tandbørste og dens opladerstation. Til selve energioverførslen benyttes induktiv kobling oftest, da det effektivt er muligt at overføre energien over en kort afstand. Induktiv kobling er en form af elektromagnetisme. Andre typer indenfor elektromagnetisme som mikrobølger kan også benyttes til trådløs energioverførsel, men da mikrobølger har en skadelig effekt på levende organismer, og induktiv kobling allerede bliver benyttet til formålet energioverførsel, så ser vi derved nærmere på denne type af elektromagnetisme.

**(MIT-artikel)** Flere fordele for induktiv kobling...

Hvorfor er det relevant at kigge på wpt i dag?

Man kan altid spørge om, hvor vi ser teknologien i fremtiden, men et andet vigtigt perspektiv er, hvor relevant trådløs energioverførsel er i dag. Vi er et progressivt folkefærd, og vi vil altid gerne anerkendes for vores projekter. Vi er nået så langt, som vi kan nå med ledninger, så det næste skridt må være at fjerne ledningerne. Med den store ændring i regeringernes klimaaftaler, skal vi også tænke på transport. Vi bliver nød til at finde en måde at få vores elbiler til at køre længere, en måde at kunne gøre det kan være ved at forbedre batterierne eller lade bilen lade op, imens den kører. Her kan man også snakke om busser, som når de holder og samler passagerer op, kan nå at lade lidt op inden den kører videre. Ved at busserne hele tiden kører rundt, er de en af de store syndere for CO2, og det kunne være en løsning på at skære ned på CO2-udledningen, vi har i dag.

Hvor ser vi at vi kan bruge det henne?

Teknologien giver base for en række nye muligheder for brug af elektriske produkter; ikke kun som gadgets, men også indenfor transportmidler og større maskineri på fabrikker. Fremadrettet ville der kunne skabes mulighed for ét samlet energisystem, der kan registrere, hvis ens elektriske produkter mangler strøm for at være fuldt opladt. Herfra kan der blive overført strøm fra en transmitter til ens elektriske produkter, uden man skal tilslutte dem en ledning i en stikkontakt. Dette kunne bl.a. implementeres i ens hjem, hvor transmittere ville kunne oplade din mobil ligegyldigt, hvor i huset man befinder sig, samtidig med at det ville kunne drive køleskabet i køkkenet og tv'et i stuen.

Set i et andet perspektiv, ville teknologien også gøre elektriske transportmidler som biler og busser mere attraktivt. Hvis transmittere bliver implementeret i vejnettet, ville man kunne oplade sin bil, mens man kører. Det vil også kunne spille godt sammen med Elon Musks planer for at have Tesla til at lave busser som kunnes skiftes ud.

En anden ting kunne være, at man ikke længere skulle trække store mængder kabler gennem fabrikshaller eller bare den almindelige husstand. Dette ville skabe en mere mobiliseret produktion, hvor fabrikkerne ikke skal tage højde for, hvordan maskinerne kan placeres, så der er strøm til hele produktionen uden at ledninger og kabler løber på tværs af fabrikshallen.

 - Hvorfor har vi valgt at undersøge mobilopladning, i forbindelse med hvor vi forventer at se teknologien senere.
 
Vi har valgt at arbejde med trådløs mobilopladning, da vi ville bruge det som et "springbræt" til måske at arbejde videre med det. En anden grund til at det også er interessant er at det er i et tidligt stadie så vi kan nå at følge med på foreste række og bedre følge udviklingen. Det giver også beder mening at starte ved den teknologi, vi har i dag, frem for at vi kaster os ud på ny og ukendt grund uden baggrundsviden for, hvordan trådløs energioverførsel hænger sammen.

\newpage
\input{historie_old}
\input{fysik_old}
\input{basis}
\subsection{WPT metoder}
De tre relevante metoder til trådløs energi, indebære induktiv magnetisk-, resonans magnetisk kobling og mikrobølger. Induktiv og resonans virker ved near-fields og mikrobølger virker ved far-ifelds.

\begin{figure}[H]
\centering
\includegraphics[scale=0.5]{Vildledning/Schematics/induktiv_resonans}
\caption{model af WPT setup}
\label{model af WPT setup}
\end{figure}

Induktiv kobling. 
WPT ved induktiv kobling fungere ved induktion over et magnetfelt mellem to spoler, hvilke genereres en strøm og spænding som ses på figure 2.4a. Magnetisk induktiv kobling sker, når den primærspole som fungere som energi senderen genererer et vekselene magnetfelt på tværs af den sekundærspole, der er energien modtageren. Dette sker inden for et området generelt mindre end bølgelængden. Denne process skaber en near-field kraft som induktiveres over den sekundære spole til en strøm/spænding, hvilke kan udnyttes af et trådløs apparatur, eksempelvis en mobil. 
Fordelen ved induktiv kobling inkludere følgende, teknologien er simple og nem at implementer ved høj effektivitet og det er sikkert for mennesker at være i nærheden. Den er dog begrænset af afstand fra transmitter og receiver, idet at den har en effektiv energioverførelse mellem nogle få millimeter til et par centimeter. Ulemper ved induktiv kobling indebære den relative korte ladeafstand, varmen der udvikles i spolerne og spolerne skal ligge meget tæt og så lige overfor hinanden som muligt for at opnå den optimale effekt.
Teknologien bliver i dag set i mobile enheder (mobiltelefoner og tablets), tandbørster, RFID-tags, induktionskomfur og betalingskort.

Resonans kobling. 
Magnetisk resonans kobling er baseret på kortvarige bølgekobling, hvilke genere og overføre en elektrisk strøm mellem to resonans spoler i et oscillerende eller varierende magnetfelt, som set på figure 2.4b. Da de to spoler er køre på samme resonans frekvens, kan der opnå en høj effektiv elektrisk energioverførelse, med meget lidt tab til  eksterne faktorer. Denne indskab gøre energioverførelsen næsten upåvirket af omgivelserne, hvilke også gør det muligt at lade selvom der er noget i mellem transmitter og receiver. 
Den klare fordel ved resonans kobling er den meget større effektive lade distance, som kan opnå en effekt på 90 procent optil 1 meter mellem transmitter og receiver, og 40 procent ud til to meter. Derudover kan en enkel transmitter lader flere receiver på samme tid så længe de operer på sammen resonans frekvens.
En ulemperne ved resonans kobling er at hver receiver kræver en dedikeret kapacitets spole, hvilke gøre det svært at gøre receiveren lille nok til mobile enheder, og det gøre det generelt mere kompliceret at implementer. 
\input{limits} %% QI er inkluderet her %%
\section{Problemformulering}

\begin{itemize}

\item Hvordan kan resonant frekvens benyttes til at forbedre afstanden hvor ved den trådløse opladning i en mobiloplader er effektiv.

\end{itemize}


	\end{folderinput}
	
\begin{folderinput} {Teori}
\chapter{Teori}
\section{Reaktans til Impedans}
Reaktans:

Ved et LCR-kredsløb, så er der ikke kun tale om modstanden fra resistoren, men hvilken modstand kapacitoren og induktoren har. Den samlede modstand for kredsløbet kan beskrives igennem kompleks impedans, som er den samlede modstand for resistoren, kapacitoren og induktoren på kompleks form. Impedansen er essentiel til beskrivelse af den faseforskydelse, der kan opstå mellem spændingen og strømstyrken i LCR-kredsløbet.

For at forstå modstanden for en kapacitor og induktor, så skal begrebet reaktans bringes på banen. Ved elektriske felter er der tale om kapacitiv reaktans, som er gældende for kapacitorer, mens der for magnetiske felter er tale om induktiv reaktans, som er gældende for induktorer. Reaktansen for henholdsvis kapacitoren og induktoren er angivet som deres modstand og optræder efter sammenhængen mellem strømstyrken og spændingen over kapacitoren og induktoren. Da de to komponenter ikke optræder ens i kredsløbet, er beregningerne for reaktansen forskellig.
\begin{equation}
Kapacitiv reaktans: X_C = - \frac{1}{j \omega C}
\end{equation}
\begin{equation}
Induktiv reaktans: X_L = j \omega L
\end{equation}

Ved ovenstående formler indgår der to konstanter C og L. C er en konstant for kapacitoren angivet som kapacitansen, mens L er en konstant for induktoren angivet som induktansen. Omega beskrives som $\omega = 2 \pi f$, hvor $2 \pi$ svarer til én svingning, mens f er frekvensen. j angiver, at reaktansen er beskrevet som et kompleks tal.

Komplekse tal bliver benyttet for bedre at kunne knytte spænding og strømstyrke sammen, så det kan afbilledes i forbindelse med den fastforskydelse, der kan forekomme. Dette forhold mellem spænding og strømstyrke kan udledes gennem omskrivning af sinus og cosinus, men for nemhedens skyld benyttes komplekse tal for at simplificere formlerne. Formlerne for reaktansen kan derved skrives om til:

\begin{equation}
Kapacitiv reaktans: X_C = - \frac{1}{j 2 \pi f C}
\end{equation}
\begin{equation}
Induktiv reaktans: X_L = j 2 \pi f L
\end{equation}
%Kilde: Reactance of capacitors and inductors

En perfekt kapasitor uden modstand ville forskyde bølgefunktionen for strømstyrken med $\frac{\pi}{2}$ frem for bølgefunktionen for spænding., hvilket svarer til en kvart svingning (1/4 frekvens. Modsat ville den perfekte induktor uden modstand sænke bølgefunktionen for strømstyrken med $\frac{\pi}{2}$ i forhold til bølgefunktionen for spænding.

Billeder af forskudte bølger (kapacitiv og induktiv reaktans)
\begin{figure}[H]
\centering
\includegraphics[scale=1]{Vildledning/Schematics/Kapacitiv_reaktans}
\caption{Kapacitiv reaktans}
\label{kreaktans}
\end{figure}
%Kilde: Capacitance in AC Circuit and Capacitive Reactance

\begin{figure}[H]
\centering
\includegraphics[scale=0.8]{Vildledning/Schematics/Induktiv_reaktans}
\caption{Induktiv reaktans}
\label{ireaktans}
\end{figure}
%Kilde: Inductive Reactance - Reactance of an Inductor

Da den komplekse impedans var LCR-kredsløbets samlede modstand, så kan det beregnes ved:
\begin{equation}
Z = R + j 2 \pi f L - \frac{1}{j 2 \pi f C}
\end{equation}
%Kilde: Complex Impedance
\newpage
\section{Magnetfelter og resonant frekvens}

Magnetisk induktiv kobling er i stand til at skabe en effektiv energioverførsel, hvor ca. 90 procent af den udsendte effekt bliver opfanget af modtageren. Dette kan samtidig gøres ved et lavt frekvensspektrum på 20 til 40kHz med en rækkevidde på op til 10cm. Magnetisk indukttiv kobling har en lille påvirkning på omkringliggende genstande, hvilket mindsker energispildet ved overførslen, men der er andre faktorer, der kan spille ind på, at effektiviteten for energioverførslen reduceres.

Placeringen for spolerne er ikke tilfældig, og det kan have en stor betydning for effektiviteten af energioverførslen, hvis spolerne ikke har direkte forbindelse. Det påvirker ikke magnetfelterne så meget, hvis ikkemagnetisk materiale er placeret mellem transistoren og modtageren. Magnetfelterne passerer gennem det ikkemagnetiske materiale, men bliver kun sænket af, at feltlinjerne magnetiserer partiklerne omkring. Til gengæld vil en forskydning af spolerne, eller at spolerne står vinklet på hinanden, reducere den optagede energi fra modtageren, da færre af de magnetiske feltlijner når modtageren. (Nicolai)

\begin{figure}[H]
	\centering
	\begin{minipage}[b]{0.48\textwidth}
	\centering
	\includegraphics[width=0.5\textwidth]{Vildledning/Schematics/forskudte_spoler} % Venstre billede
	\end{minipage}
	\hfill
	\begin{minipage}[b]{0.48\textwidth}
	\centering
	\includegraphics[width=0.5\textwidth]{Vildledning/Schematics/vinklet_spoler} % Højre billede
	\end{minipage}
	\\ % Figurtekster og labels
	\begin{minipage}[t]{0.48\textwidth}
	\caption{Forskudte spoler} % Venstre figurtekst og label
	\label{fspoler}
	\end{minipage}
	\hfill
	\begin{minipage}[t]{0.48\textwidth}
	\caption{Vinklet spoler} % Højre figurtekst og label
	\label{vspoler}
	\end{minipage}
\end{figure}

Resonant frekvens:

Når der løber en strøm gennem et elektrisk kredsløb med vekselstrøm, så er der tilsvarende en frekvens for, hvor hurtigt strømmen skifter retning i systemet. Ved induktiv kobling er frekvensen med til at bestemme outputtet for det elektriske kredsløb. Den specifikke frekvens for et kredsløb, der giver det største output, kaldes for den resonante frekvens. For at energioverførslen mellem transistoren og modtageren skal være mest effektiv, så skal begge kredsløb have en ens resonant frekvens. På denne måde bliver effekten udsendt med størst mulig output, og modtageren kan opfange effekten på bedst mulig vis.
\end{folderinput}

	\begin{folderinput}{Lab}
\chapter{Forsøg}
\section{Forsøgsbeskrivelse}
Hej med dig
	\end{folderinput}

	\begin{folderinput}{Beregninger}
\chapter{Forsøgsberegninger}

\begin{itemize}
\item \textcolor{red}{Her kommer der til at være en intro til hvad vi vil have ud af beregningerne.}
\end{itemize}

\section{LC kreds}

\begin{equation} \label{angfreq}
	\omega = \frac{1}{\sqrt{LC}}
\end{equation}

Denne ligning relaterer: frekvens, kapacitans og induktans.

Der løses for L da denne er besværlig at måle sammenlignet med den aflæste kapacitans:

\begin{equation}
	\omega = \frac{1}{\sqrt{LC}} 
\end{equation}

Da $\omega = 2\pi f$ er kendt kan der løses for L:

	\end{folderinput}

	\begin{folderinput}{Vildledning}
\chapter{Forsøg 1} \label{bilag:forsg1}

P1 - Gruppe C-16a - 07-Nov-16

\begin{figure}[htbp]
\centering
\includegraphics[width=1\textwidth]{Vildledning/Schematics/Eks1_LCR.png}
\caption{Foreslåede eksperiment kredsløb. R-kreds - CR-kreds - LCR-kreds}
\label{fig:Eks1}
\end{figure}
\newpage

\section{Hvad vil vi opstille?}
\begin{itemize}
\item R-kreds
\item CR-kreds
\item LCR-kreds
\end{itemize}
Til alle opstillingerne vil der i starten benyttes et oscilloskop som kilde, sat til vekselspænding ved 5 V, 50Hz.

De tre opsætninger benyttes som reference punkter til videre eksperimentring.  
\section{R-kreds}
Til at starte med vil vi opstille, det måske aller simpleste kredsløb, ved bare at sende en vekselspænding over en resistor og måle spændingsforskellen og strømmen over det.

Herefter vha. $U=R\cdot I$ kan der så undersøges om den aflæste værdi af resistoren er den samme som den målte.
\section{CR-kreds}
Opstillingen her er ens med R-kredsen, uden ampere-meter, bare at der nu er sat en kapacitator på og volt-meteret er flyttet hen over kapacitatoren.

Herefter ses der på om der er sket en ændring i spændingsforskellen over kapacitatoren, sammenlignet med den mængde spænding der er tilført systemet.
\section{LCR-kreds}
Dette er den vigtigste kreds der undersøges, da der nu er lavet et loop ved at sætte en spole i kredsløbet. Volt-meteret der benyttes her skulle gerne være et der kan tegne grafer, specifikt $(U,t)$.

Kredsløbet er en forlængelse af de to andre. Volt-meteret er nu bare flyttet til hen over spolen, da dette er den komponent med størst relevans.

Her undersøges der den spændingskurve der kan tegnes over spolen. Denne kan herefter sammenlignes med kurven i givet fra en simulering i Plecs. Dette er selvfølgelig kun relevant hvis det er muligt at komme tæt på virkelige værdier i programmet.

	\end{folderinput}
	
	
\bibliography{bib/bibfil} %% Litteraturlisten indsættes her %%


\appendix
\clearforchapter
\phantomsection
\pdfbookmark[0]{Appendiks}{appendiks}
\chapter{Bilag}
Hej med dig

%\chapter{PV-fremadrettet}
Nu da vi har lavet vores første forsøg, hvor vi har set på, hvordan spændingen skifter over en modstand, kapacitator og spole ved ændring på frekvensen, kan vi nu gå videre til vores forsøg om WPT. Vi fik chancen for at omlægge den teori, vi har læst om, ud til en forsøgsopstilling, hvor vi kunne aflæse forskellige resultater, der giver mulighed for videre arbejde. Det har ikke lokket os fra at skulle arbejde videre med projektet, snarere det modsatte. Vi har fået nyt blod på tanden og glæder os til de afsluttende forsøg. Forsøgene skal benyttes til databehandlingen, så vi kan sammenligne resultaterne, så vi har konkrete tal at arbejde ud fra, når vi skal videre med problemløsningen. Vores strategi for fremtiden er at vi vil skriftes til at lave forsøg og dobbelttjekke hinandens forsøgsopstillinger. Når den ene del af gruppen er i laboratorierne vil den anden del af gruppen ihærdigt skriver videre, laver beregninger, skematics osv. til rapporten. Det er vores målsætning at holde møde hver fredag, så vi kan holde trit med vores tidsplan, og at hvis vi kommer bag ud, er det muligt for de andre i gruppen at træde til og hjælpe et medlem. Dette er ingen individuel rapport, og vi skal alle sammen kunne stå inde for rapporten. Ergo er det vigtigt, at man hjælper hinanden, for hvis der er en som taber, taber hele gruppen. Det er lige meget hvis skyld, det er, for vi er alle i samme båd.

For at holde gejsten oppe i gruppen sørger vi for at holde hinanden i nakken, så der ikke er nogen, der får problemer og falder bagud i forhold til gruppen. Projektet er en samlet indsats, så det er vigtigt at få lavet opsamlinger i gruppen, så alle kan følge med, og alle har samme udgangspunkt for senere arbejde. Derudover diskuterer vi om, hvad der interesserer os ved projektet, og hvad vi gerne vil have ud af det færdige produkt, så det holder os fast på et endeligt mål. Hvis vi arbejder med det, vi finder interesse for, er det lettere at holde sig fokuseret på arbejdsopgaverne, og man får lyst til at gøre en god indsats (ikke kun for en selv, men også for gruppen som helhed).

Som gruppe bliver vi enige om, hvilke arbejdsområder vi skal undersøge og skrive om. Herefter inddeler vi os i mindre grupper for hver arbejdsopgave (2-3 mand pr. gruppe). Dette gør, at vi hurtigt kan få indsamlet brugbar viden og udført en stort stykke arbejde, men at vi samtidig ikke står alene med en opgave. Det at være i små grupper gør også, at man kan få flere input og idéer til, hvad man eventuelt kan bringe ind over opgaven, hvilket gør at projektet bliver mere nuanceret. For at holde styr på, hvor langt hver gruppe er, og hvad de hver især har skrevet, så holder vi jævnligt møder til at opsummere processen.

Da vi er igennem problemanalysen, så har vi fået dannet os en god grundviden om projektet, som vi videre kan benytte til forsøg, modeller og projektløsningen senere hen. Dette betyder også, at vi nu skal til at specificere os på enkelte dele af projektet, så vi får indsnævret vores undersøgelser.

\end{document}