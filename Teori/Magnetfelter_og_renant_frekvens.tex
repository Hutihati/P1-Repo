\section{Magnetfelter og resonant frekvens}

Magnetisk induktiv kobling er i stand til at skabe en effektiv energioverførsel, hvor ca. 90 procent af den udsendte effekt bliver opfanget af modtageren. Dette kan samtidig gøres ved et lavt frekvensspektrum på 20 til 40kHz med en rækkevidde på op til 10cm. Magnetisk indukttiv kobling har en lille påvirkning på omkringliggende genstande, hvilket mindsker energispildet ved overførslen, men der er andre faktorer, der kan spille ind på, at effektiviteten for energioverførslen reduceres.

Placeringen for spolerne er ikke tilfældig, og det kan have en stor betydning for effektiviteten af energioverførslen, hvis spolerne ikke har direkte forbindelse. Det påvirker ikke magnetfelterne så meget, hvis ikkemagnetisk materiale er placeret mellem transistoren og modtageren. Magnetfelterne passerer gennem det ikkemagnetiske materiale, men bliver kun sænket af, at feltlinjerne magnetiserer partiklerne omkring. Til gengæld vil en forskydning af spolerne, eller at spolerne står vinklet på hinanden, reducere den optagede energi fra modtageren, da færre af de magnetiske feltlijner når modtageren. (Nicolai)

\begin{figure}[H]
	\centering
	\begin{minipage}[b]{0.48\textwidth}
	\centering
	\includegraphics[width=0.5\textwidth]{Vildledning/Schematics/forskudte_spoler} % Venstre billede
	\end{minipage}
	\hfill
	\begin{minipage}[b]{0.48\textwidth}
	\centering
	\includegraphics[width=0.5\textwidth]{Vildledning/Schematics/vinklet_spoler} % Højre billede
	\end{minipage}
	\\ % Figurtekster og labels
	\begin{minipage}[t]{0.48\textwidth}
	\caption{Forskudte spoler} % Venstre figurtekst og label
	\label{fspoler}
	\end{minipage}
	\hfill
	\begin{minipage}[t]{0.48\textwidth}
	\caption{Vinklet spoler} % Højre figurtekst og label
	\label{vspoler}
	\end{minipage}
\end{figure}

Resonant frekvens:

Når der løber en strøm gennem et elektrisk kredsløb med vekselstrøm, så er der tilsvarende en frekvens for, hvor hurtigt strømmen skifter retning i systemet. Ved induktiv kobling er frekvensen med til at bestemme outputtet for det elektriske kredsløb. Den specifikke frekvens for et kredsløb, der giver det største output, kaldes for den resonante frekvens. For at energioverførslen mellem transistoren og modtageren skal være mest effektiv, så skal begge kredsløb have en ens resonant frekvens. På denne måde bliver effekten udsendt med størst mulig output, og modtageren kan opfange effekten på bedst mulig vis.