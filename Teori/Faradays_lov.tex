\section{Faraday's lov}

Faraday's lov beskriver induktionen af elektricitet, ved hjælp af magnetisme. Herved omhandler det den magnetiske flux, i stedet for den elektriske flux, som bliver brugt ved Gauss's lov. Formlen for magnetisk flux er ens med formlen for den elektriske flux, dog hvor det elektriske felt er byttet ud med det magnetiske felt: $\Phi_B = \int \vec{B} \cdot \vec{dA}$

En induceret strøm opstår ikke fra den magnetiske flux alene, men ved en ændring i den magnetiske flux. Dette betyder, at der bliver induceret spænding, dvs. hvis der sker en ændring af magnetfeltets styrke, den påvirkede overflades størrelse eller vinklen for, hvordan det magnetiske felt går gennem den pågældende overflade.

Faraday benytter den magnetiske flux til, at beskrive den inducerede spænding ved \cite{fysikbog}:

\begin{equation} \label{faraday}
\centerline{$\varepsilon = -1 \cdot \frac{d \Phi_B}{dt}$}
\end{equation}

Ændringen af den magnetiske flux forekommer modsat af den inducerede spænding, så derfor ganges en faktor $-1$ på det differentierede udtryk af den magnetiske flux. Den magnetiske flux kan også beskrives som: $\vec{B} \cdot \vec{A}$ eller $B \cdot A \cdot cos(\theta)$.
\newpage