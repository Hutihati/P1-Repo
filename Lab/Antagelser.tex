\section{Antagelser forud for forsøg}

Forsøg med LC-kredsløb:

Spændingen over kapacitoren vil først stige og derefter falde, når frekvensen for systemet øges. Før den resonants frekvens opnåes vil spændingen stige langsomt. Nær den resonante frekvens vil der forekomme et udslag, hvor toppunktet for spændingen vil opstå. Efter toppunktet falder spændingen kraftigt, hvorefter faldet stilner af, så spændingen kun falder en smule ved stadig at øge frekvensen. Hvis den resonante frekvens ikke opnåes, så vil der ske en lille stigning i spændingsfaldet over kapacitoren.

Forsøg med LCR-kredsløb (Modstand):

Ved måling over modstanden i kredsløbet vil spændingen over modstanden stige i takt med frekvensen. Derved vil en øget frekvens tilsvarende resultere i et øget spændingsfald over modstanden.

Forsøg med LCR-kredsløb (Induktor):

Frekvensen er antal svingninger igennem systemet, og jo flere svingninger des større magnetfelt vil blive dannet ved induktoren. Spændingsfaldet over induktoren vil derved stige i takt med, at frekvensen stiger. Dette vil forekomme som en liniær rekretion, hvor frekvensen og spændingsfaldet stiger i takt med hinanden.