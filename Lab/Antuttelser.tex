Antagelser:

Forsøg med LC-kreds:

Ved måling over kapacitoren, når der sker en stigning i frekvens, så vil der ske en stigning i spændingsfaldet over kapacitoren. Denne stigning vil kun øge en smule, til der nærmes den resonante frekvens for LC-kredsløbet. Spændingsfaldet over kapacitoren vil begynde at stige kraftigt nær den resonante frekvens, hvor den vil nå sit toppounkt. Herefter vil spændingen igen falde, hvis frekvensen overstiger den resonante frekvens. Hvis den resonante frekvens ikke opnåes, så vil der ske en lille stigning i spændingsfaldet over kapacitoren.

Forsøg med LCR-kreds (Resistor):
Resistoren ændrer ikke værdi gennem forsøget, men har en fast resistens i kredsløbet, hvorved spændingsfaldet burde holdes stabilt. Derved sker der ikke nogen ændring i spændingsfaldet over spolen, når frekvensen øges, og der vil forekomme en liniær rekretion for spændingsfaldet i forhold til frekvens, hvor spændingsfaldet forbliver uændret.

Forsøg med LCR-kreds (Induktor):
Frekvensen er antal svingninger igennem systemet, og jo flere svingninger des større magnetfelt vil blive dannet ved induktoren. Spændingsfaldet over induktoren vil derved stige i takt med, at frekvensen stiger. Dette vil forekomme som en liniær rekretion, hvor frekvensen og spændingsfaldet stiger i takt med hinanden.