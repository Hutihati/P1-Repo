\section{Forsøg 1 - Kredsløb}

\subsection{Forsøgsbeskrivelse}

\subsubsection{Formål}
Formålet med disse forsøg er at for en forståelse af, hvordan de forskellige kredsløb fungere. Derudover bliver der også dannet en baggrundsviden af, hvordan det teoretiske er i kredsløbet. Måledataenes vær-dier kan senere bruges til, at beregne på de forskellige komponenter i kredsløbet. Ud fra måledataene kan der findes en induktans på spolen, som senere kan benyttes til værdi i Plecs, så vores forsøg beskrivelser også kan simuleres, derudover benyttes måledataene også til, at regne på andre forskellige komponenter. Alt dette findes i afsnittet med beregninger.

\subsubsection{LC serie kredsløb}
%\textbf{LC Serie kredsløb}

Komponenter:

\begin{itemize}
\item Generator $(50\, \Omega$ modstand)
\item Kapacitor $( 0,1\, \mu F)$
\item Spole/Inductor $(1600$ vindinger)
\item Oscilloskop
\end{itemize}

Opstilling:

\begin{figure}[H]
\centering
\includegraphics[scale=1]{Vildledning/Schematics/Kredslb/LC_Serie}
\caption{LC Serie Kredsløb}
\label{lcserie}
\end{figure}

Figur \ref{lcserie} viser, hvordan forsøgsopstillingen er bygget op. Forsøget består af en generator, spole og en kapacitor, hvor der sat et oscilloskop til.

\subsubsection{LC parallel kredsløb}

Komponenter:

\begin{itemize}
\item Generator $(600\, \Omega$ modstand)
\item Kapacitor $( 0,1\, \mu F)$
\item Spole/Inductor $(1600$ vindinger)
\item Oscilloskop
\end{itemize}

Opstilling:

\begin{figure}[H]
\centering
\includegraphics[scale=1]{Vildledning/Schematics/Kredslb/LC_Parallel}
\caption{LC Parallel Kredsløb}
\label{lcparallel}
\end{figure}

Figur \ref{lcparallel} viser forsøgsopstilling af et LC parallel kredsløb, hvor hele kredsløbet sidder parallelt. Kredsløbet er bestående af en generator, en spole og en kapacitor, hvor et oscilloskop er sat til.

\subsubsection{LCR målt over spolen}

Komponenter:

\begin{itemize}
\item Generator $(50\, \Omega$ modstand)
\item Kapacitor $( 0,1\, \mu F)$
\item Spole/Inductor $(1600$ vindinger)
\item Modstand/Resistance $(500\, \Omega)$
\item Oscilloskop
\end{itemize}

Opstilling:

\begin{figure}[H]
\centering
\includegraphics[scale=1]{Vildledning/Schematics/Kredslb/LCR_spole}
\caption{LCR-kredsløb målt over spolen}
\label{lcrspole}
\end{figure}

Figur \ref{lcrspole} viser forsøgsopstilling af et LCR-kredsløb. I dette forsøg bliver der målt over spolen, altså oscilloskopet er sat over spolen. Kredsløbet består af en generator, en spole, en kapacitor og en modstand.

\subsubsection{LCR målt over modstand}

Komponenter:

\begin{itemize}
\item Generator $(50\, \Omega$ modstand)
\item Kapacitor $( 0,1\, \mu F)$
\item Spole/Inductor $(1600$ vindinger)
\item Modstand/Resistance $(500\, \Omega)$
\item Oscilloskop
\end{itemize}


Opstilling:

\begin{figure}[H]
\centering
\includegraphics[scale=1]{Vildledning/Schematics/Kredslb/LCR_modstand}
\caption{LCR-kredsløb målt over spolen}
\label{lcrmodstand}
\end{figure}

Figur \ref{lcrmodstand} viser forsøgsopstilling af et LCR-kredsløb. Modsat tidligere forsøg, så bliver der målt over modstanden. Kredsløbet er bestående af samme komponenter som det tidligere forsøg.

\subsubsection{Fremgangsmåde}

Udførelsen af forsøgende er næsten ens, de variere ikke ret meget, forsøgsopstillingen bliver ændret fra forsøgs til forsøg, hvor der skiftets ud i komponenter, så det passer med det rigtige kredsløb.

Først opstilles kredsløbet, som det er vist på figurene \ref{lcserie}, \ref{lcparallel}, \ref{lcrspole} og \ref{lcrmodstand}. Derefter skal oscilloskopet indstilles, så det måler de rigtige data, det indstilles til at måle en: peak-to-peak. Så er forsøget klar til at starte, der tændes for generatoren, og dens frekvens indstilles til den ønskede værdi, og peak-to-peak værdien noteres. Dette gøres med forskellige frekvenser, så der fås forskellige data, som senere kan plottes ind i en tabel, og der kan derved laves en graf for det.

\newpage