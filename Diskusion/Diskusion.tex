Wireless power transfer er ikke så veludviklet i dag, hvilket ikke gør det optimalt at implementere i produkter, før der er sket væsentlige forbedringer. Fra teknologiens standpunkt så kan det ikke betale sig at benytte trådløs opladning til afstande på mere end 10cm ved effetive transmittere og modtagere. Almindelige trådløse opladere i dag har brug for, at produktet skal ligge tæt op ad transmitteren for at få en stabil opladning. Ved en gennemgang af forsøget med den trådløse oplader fra IKEA, viser det sig, at der kan opstå usikkerheder ved en afstand på $0,5 cm$ og ved større afstande mellem transmitter og modtager, går signalet tabt i omgivelserne. Derudover ses der en sammenhæng mellem afstanden og effektiviteten på opladningen, hvor en forøgelse i afstanden mellem transmitter og modtager mindsker den overførte strøm til produktet.

Modsat bliver opladningen af ens mobiltelefon mere mobiliseret, da opladerkablet ikke skal trækkes fra en stikkontak og direkte til mobiltelefonen. Ved igen at se på opladeren fra IKEA, så er den bygget til at kunne indsættes i bordpladen på et spisebord eller lignende, så telefonen bare skal placeres på opladeren i bordpladen i stedet for at tilslutte den til et kabel. Ved at kunne overføre energien til mobiltelefoner trådløst, så sker der også en ændring på selve mobilen, da indgangen til opladerkablet bliver overføldigt. Dette kan bl.a. have den fordel, at telefonen lettere kan gøres vandtæt. Integrationen af trådløs energioverførsel til telefonen kan gøres simpelt ved indsættelse af et LCR-kredsløb, der skal have en tilhørende transmitter, som har samme resonante frekvens, hvilket gør overgangen fra de traditionelle kabler til trådløs energioverførsel enkel.

Der er stor forskel på, hvor effektiv energioverførslen er i forhold, hvilken frekvens transmitterens og modtagerens system kører på. For en optimal energioverførsel, så skal begge kredsløb have en høj effektivitet ved samme frekvens. Herfra er indstillingen af den operative frekvens afhængig af, hvor stor modstand, der er indkorporeret i systemet alt efter hvilket produkt, der skal oplades. Til mobilopladning som ikke behøver en stor strømstyrke overført, så kan der med fordel benyttes en stor modstand, hvorved frekvensområdet for en stabil overførsel bliver bredt. Dette gør, at der kan ske udsvingninger i frekvensen mellem transmitter og modtager, uden strømstyrken svinger betydeligt.

Hvor modstanden er med til at afgøre hvor stort en frekvensområde, der kan arbejdes indenfor, så er den rigtige frekvens betydende for, hvornår strømstyrken stiger og falder. LCR-kredsløb har en resonant frekvens, hvor det største output forekommer. For at benytte resonant kobling mellem transmitter og modtager, så skal begge kredsløb have en ens resonant frekvens. Resonant kobling vil sikre den mest optimale overførsel af energi, men hvis de to kredsløb ikke en ens resonant frekvens, kan der ske store ændringer i hvor meget strøm, der bliver overført.