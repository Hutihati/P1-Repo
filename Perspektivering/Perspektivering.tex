\chapter{Perspektivering}

Trådløs energioverførsel er ikke et nyt koncept, men det er først i nyere tid, hvor det er begyndt at vinde indtog. Flere og flere får øjnene op for de forskellige anvendelser, der ligger i teknologien trådløs energioverførsel, og det er et stort forskningsemne. På trods af den stigende interesse er overførslen af energi ikke effektiv, og meget af energien går tabt ved overførsel over korte afstande. For videre at skabe rammer for interesse og fokuspunkter for udvikling af teknologien, kan det være godt at se på område, hvor det kunne være nyttigt at benytte teknologien.

Opladning af elektroniske gadgets og hjælpemidler:
 	
Trådløs energiopladning giver grundlag for en mobiliseret dagligdag, hvor elektriske produkter ikke længere er afhængige af at skulle tilsluttes en stikkontakt, når de er ved at løbe tør for strøm. På den kortere bane kan der fokuseres på opladning af mindre elektriske gadgets som f.eks. mobiltelefoner eller pacemakers. Når teknologien er udviklet nok til, at der kan overføres en svag strøm over lange distancer, så kunne der placeres rutere (lignende rutere for netværksforbindelse wifi) rundt omkring i bylandskabet, så ens elektriske apparater kunne oplades på vej til og fra arbejde. På mindre skala kunne det være, at busser, indkøbscentre eller almindelige husstande kunne udsende energi trådløst til de elektroniske produkter, der befinder sig indenfor rækkevidde. Herved bliver batteriers levetid mindre betydende, da de passivt kan blive opladt nær rutere eller lignende, så længe det elektriske apparat er inden for ruternes radius. Dette gør, at ledninger bliver overflødige for mindre produkter. Opladning af gadgets ved hjælp af trådløs energioverførsel kan ses som en luksus, men det kan også være med til at afhjælpe personer, der benytter elektrisk apparatur for egen sikkerheds skyld. Herunder kan der være tale om opladning af pacemakers eller måleinstrumenter til personer med sukkersyge, som er afhængige af, at produkterne er opladt og klar til brug.

Kommunikation mellem produkt og transmitter:
 	
Som udbygning af konceptet med at oplade mobiltelefoner og andre gadgets, så skal der også tages højde for, hvornår og hvor meget transmitteren skal udsende af energi til de enkelte produkter. Transmitteren kan i princippet fortsætte med at overføre energi til produkterne, men overførsel af unødvendig elektricitet vil ikke være økonomisk favorabelt. Derudover vil det overbelaste og skade batterierne, hvis det konstant tilføres strøm, hvis produktet ikke har indbygget en begrænsning for opladningen. En idé til at forhindre overførsel af energi, som ellers ville gå til spilde, kunne være, hvis produkter kombatible til trådløs opladning kunne kommunikere med transistoren. Herved kunne transistoren opfange data omkring apparaternes batteriniveau og kun udsende strøm mellem en bestemt batteriprocent.

Implementering i bygninger og fabrikshaller:
 	
Som WPT udvikler sig, bliver det nemmere at overføre elektricitet og større mængder af det. Opladning af mobiltelefoner er et springbræt for udviklingen og kan derefter tages til nye højder. På et tidspunkt ville det ikke være nødvendigt at trække kabler gennem hele bygningskonstruktioner, men kun hen til, hvor transmitterne for den trådløse energioverførsel skal placeres. Organisatorisk vil dette være banebryden bare at indføre i almindelige husstande. Hvis køleskabet, komfuret og TV'et blev holdt igang vha. trådløs energioverførsel, så skal det ikke placeres nær stikkontakter, men det kan placeres, hvor det ønskes, så længe det befinder sig indenfor transmitterens rækkevidde. Det gør det lettere at flytte rundt på elektriske apparater, og de kan placeres i forhold til mindre pladsspild, da der ikke skal trækkes stikdåser til opkoblingen. Ved husstanden har det måske ikke den helt store betydning for at udnytte pladsen, men for værksteder eller fabrikker, kan det påvirke meget for produktionen for, hvordan maskiner drevet af elektricitet er placeret i forhold til hinanden. Hvis en udsender er ophængt under loftet, så den kan drive de forskellige maskiner, så skal der ikke trækkes kabler på tværs af fabrikshallen, og det giver større frihed for placeringen af maskinerne. Dette kan have stor indvirkning på fabrikkens effektivitet, da maskinerne bedre kan opstilles i forhold til produktionens behov.

Opkobling til trådløs energioverførsel:
 	
Ved indførelse af WPT i dagligdagsbilledet stiller det et spørgsmål til, om alle elektriske apparater har forbindelse til hver transmitter, eller om produkterne skal kobles med den enkelte sender. Ved offentlige områder kan det være der benyttes åbne netværker, som ved områder med gratis wifi, men ved private områder og husstande kan det give komplikationer, hvis alle har adgang til opladning af f.eks. mobiltelefoner. Derved kunne det med fordel være godt at kunne sætte adgangskode eller anden form for sikkerhed på signalet fra transmitterne, så der kan kontrolleres, hvem der har adgang. Hvis der stadig skal betales for den strøm, der bliver overført fra transmitteren, så kan det gå ud over de enkelte boliger, hvis folk på gaden kan få forbindelse til deres netværk og forbruge strøm.
 
Elektriske transportmidler:
 	
Der forekommer flere og flere elektriske produkter på markedet, og et punkt som udvikler sig hurtigt i dag er elektriske køretøjer som elbiler. Dette er også et punkt, hvor trådløs energioverførsel kan benyttes. Elektriske biler skal kobles til opladerstationer for at blive genopladt, men WPT kan gøre det muligt at oplade bilerne, mens de befinder sig på vejene. Især elektriske busser er på tale i dag, hvor der bliver set på, om busserne kan oplades omkring stoppesteder eller andre pladser, hvor bussen holder stille. Et andet koncept er at integrerer transmitterne i selve vejnettet, hvor vejene er inddelt i sektioner, som kan tændes separat. Herfra registreres busserne, hvor de er, hvorefter sektionen, de befinder sig på, begynder at overføre energi til bussen. På denne måde bliver bussen opladt, mens den kører gennem byen. Denne benyttelse af trådløs energioverførsel kan gøre det mere attraktivt at investere i elektriske køretøjer i stedet for transportmidler, der kører på fossilt brændsel.
 	
Energiindustrien:
 	
Set i det store perspektiv, når WPT kan overføre store mængder energi over lange afstande, så kan det begynde at benyttes ved energisektoren. Her kan der f.eks. ses på havvindmølleparker. Det kræver et kæmpe arbejde for at anlægge havvindmølleparker bl.a. i forbindelse med opkoblingen til fastlandet. Der skal trækkes kilometervis af kabler på havbunden samt transformerstationer for at holde strømmen stabil. Derudover er det et logistisk problem, hvis der skal foretages reparationer på kablerne. Alt tal sammen er også set som en stor udgift og arbejdskraft, som kan mindskes betydeligt, når WPT bliver udviklet nok til at implementeres i stedet for. Vindmøllerne kunne forbindes et en eller få transmittere ved selve parken, som ved induktiv kobling kunne overføre strømmen til transformerstationer på fastlandet, som derfra ville kunne føre strømmen videre til forbrugerne.

Delkonklussion:

Trådløs energioverførsel er ikke så veludviklet i dag, men det er et stort forskningsemne, som hele tiden forandre sig. Senere vil der optræde et væld af muligheder for brugen af teknologien, som dækker over et bredt spektrum fra mobilopladning i ens egen husstand til overførsel af energi fra havvindmølleparker til fastlandet. Med hjælp fra WPT vil transportsektoren kunne ændre sig til, at elektriske køretøjer får en længere rækkevidde og kan blive mere favorabel i bymiljøet. Der optræder dog andre problemstillinger i forbindelse med, at teknologien forbedres, som skal tages højde for på længere sigt.