\chapter{Forsøgsberegninger}

\begin{itemize}
\item \textcolor{red}{Her kommer der til at være en intro til hvad vi vil have ud af beregningerne.}
\end{itemize}

\section{LC kreds}

\begin{equation} \label{angfreq}
	\omega = \frac{1}{\sqrt{LC}}
\end{equation}

Denne ligning for vinkelbestemt frekvens relaterer: frekvens, kapacitans og induktans.

Der løses for L da denne er besværlig at måle sammenlignet med den aflæste kapacitans. $\omega = 2\pi f$ så denne indsættes samtidigt:

\begin{equation}
	\omega = \frac{1}{\sqrt{LC}} \Leftrightarrow L = \frac{1}{\omega^2 \cdot C} = \frac{1}{4\pi^2 f^2 \cdot C} = \frac{1}{4\cdot \pi^2 f^2\cdot 0.1\mu H}
\end{equation}

\begin{itemize}
\item \textcolor{red}{frekvensens her skulle nok være resonant frekvens $\omega_0$ for kredsløbet, denne er vi nød til at finde først måske ved hjælp af $I_p$ s. 615}
\end{itemize}