\section{Grundlæggende formler}

Induktiv kobling:

For at få kendskab til, hvordan trådløs energioverførsel opstår, så skal der beskrives, hvordan elektrisk flux opfører sig, hvilket bliver beskrevet gennem Gauss's lov. Derudover ses der på, hvordan sammenhængen mellem magnetisme og elektricitet er beskrevet gennem Faraday's og Ampère's lov. Til sidst ses der på, hvordan der kan dannes symmetri imellem dem, dette gøres ved Maxwell's lov.

Elektrisk flux:

Elektrisk flux er beskrevet gennem Gauss's lov, som involverer de elektriske feltlinjer samt det areal, som feltlinjerne påvirker. Flux begrebet i sig selv er betegnet som det "flow" af en given substans (f.eks. vand, luft eller elektroner), der løber gennem et givet areal.

Formlen for flux angiver det elektriske felt ganget med arealet, som det bevæger sig igennem: $\Phi = E \cdot A$. Da indfaldsvinklen for det elektriske felt også har betydning for fluxen, så gælder det: $\vec{E} \cdot \vec{A}$ i stedet for, hvilket omskrives til: $\Phi = E \cdot A \cdot cos(\theta)$. (Se figur \ref{vinkelflux})

\begin{figure}[H]
\centering
\includegraphics[scale=0.5]{Vildledning/Schematics/Vinkelflux}
\caption{Vinkelflux}
\label{vinkelflux}
\end{figure}

Gauss's lov angiver, hvordan den elektriske flux gennem et bestemt areal kan beregnes, denne er givet ved vektoren $A$ ganget med faktoren $d$, hvilket deler arealet op i brudstykker. Herefter tages integralet af formlen for flux, hvilket giver følgende udtryk: $\Phi = \int \vec{E} \cdot d \vec{A}$, som videre kan skrives som: $\Phi = \int E \cdot dA \cdot cos(\theta)$. Herefter tager Gauss relation til det cirkulære felt omkring en positiv ladning. $A$ bliver i denne sammenhæng formlen for en kugles overflade $4 \pi r^2$, mens integralet ophæves, da det er hele overfladen. Altså der fås: $\Phi = E \cdot 4 \pi r^2$

Det elektriske felt $E$ er også angivet til at være $\frac{kq}{r^2}$. Derved bliver den elektriske flux: $\Phi = \frac{kq}{r^2} \cdot 4 \pi r^2 = 4 \pi k q$. $k$ er derudover defineret som $\frac{1}{4 \pi \epsilon_0}$, hvilket indsættes i formlen for elektrisk flux: $\frac{4 \pi q}{4 \pi \epsilon_0} = \frac{q}{\epsilon_0}$.

$q$ angiver den omkransede ladning for en lukket overflade. Derved kan Gauss's lov skrives som\cite[Kap. 21]{fysikbog}:

\begin{equation} \label{gauss}
\centerline{$\oint \vec{E} \cdot d \vec{A} = \frac{q}{\epsilon_0}$} 
\end{equation}


Elektromagnetisme:

Ampère's lov:

Ampère's lov beskriver relationen mellem magnetiske feltstyrker og størrelsen af en jævn strøm, gennem en ledning givet over længden $l$. Ampère tager udgangspunkt i centrumet af ledningens tværsnit, som følger magnetfeltet, som omkredser ledningen. Her er magnetfeltets styrke defineret ved vektoren: $\vec{B}$, og et definerede linjestykke af magnetfeltets længde, angives som: $\vec{dl}$. For at beregne den jævne strøm gennem ledningen, skal der tages integralet af de to vektorer prikket sammen. Herved beskrives Ampére's lov \cite{fysikbog}:

\begin{equation} \label{ampere}
\centerline{$\oint \vec{B} \cdot \vec{dl} = \mu_0 I$}
\end{equation}

Vektor $\vec{B}$ er angivet ved $\frac{\mu_0 I}{2 \pi r}$, da magnetfeltet er cirkelformet. Derudover er det lukkede integral af $\vec{dl}$ den totale længde af cirkelperiferien angivet ved $2 \pi r$. Produktet mellem disse vil dermed blive $\mu_0 I$, som er angivet på højre side af Ampére's lov.

Faraday's lov:

Faraday's lov beskriver induktionen af elektricitet, ved hjælp af magnetisme. Herved omhandler det den magnetiske flux, i stedet for den elektriske flux, som bliver brugt ved Gauss's lov. Formlen for magnetisk flux er ens med formlen for den elektriske flux, dog hvor det elektriske felt er byttet ud med det magnetiske felt: $\Phi_B = \int \vec{B} \cdot \vec{dA}$

En induceret strøm opstår ikke fra den magnetiske flux alene, men ved en ændring i den magnetiske flux. Dette betyder, at der bliver induceret spænding, dvs. hvis der sker en ændring af magnetfeltets styrke, den påvirkede overflades størrelse eller vinklen for, hvordan det magnetiske felt går gennem den pågældende overflade.

Faraday benytter den magnetiske flux til, at beskrive den inducerede spænding ved \cite{fysikbog}:

\begin{equation} \label{faraday}
\centerline{$\varepsilon = -1 \cdot \frac{d \Phi_B}{dt}$}
\end{equation}

Ændringen af den magnetiske flux forekommer modsat af den inducerede spænding, så derfor ganges en faktor $-1$ på det differentierede udtryk af den magnetiske flux. Den magnetiske flux kan også beskrives som: $\vec{B} \cdot \vec{A}$ eller $B \cdot A \cdot cos(\theta)$.

Maxwell's ligninger for vekselstrøm:

Trådløs opladning går ud på, at omdanne elektrisk flux til magnetisk flux gennem en spole ved transmitteren, hvorefter den magnetiske flux igen skal omdannes til en elektrisk flux ved modtageren. For at beskrive hvordan elektriske felter omdannes til magnetisk flux, så skal Ampére's lov benyttes. Herefter kan overgangen fra magnetfelt til elektrisk flux beskrives gennem Faraday's lov. Til sidst kan der ses på Maxwell's ligninger, som bygger videre på Ampère's og Faraday's love, der kan så dannes en sammenhæng mellem dem.

Maxwell indså, at der måtte foretages modifikationer for Ampére's lov, hvis der skulle kunne skabes symmetri med Faraday's lov. Ved Maxwell's ligninger er Faraday's lov opgivet som, det lukkede linjeintegral af det magnetiske felt, som er lig det negative differentiale af den magnetiske flux i forhold til tid: $\oint \vec{E} \cdot \vec{dl} = -1 \cdot \frac{d \Phi_B}{dt}$.

Herefter kan der tages et blik på Maxwell's modificerede udgave af Ampère's lov. Maxwell har her udbygget formlen, så der skabes en symmetri med Faraday's lov. Derved bliver Ampère's lov omskrevet til, at det lukkede linjeintegral af det magnetiske felt er lig den elektriske spænding, lagt sammen med differentialet af den elektriske flux i forhold til tiden, hvorpå der er ganget en faktor bestående af produktet mellem permeabilitetskonstanten og permittivitetskonstanten \cite{fysikbog}: 

\begin{equation} \label{maxwell}
\centerline{$\oint \vec{B} \cdot \vec{dl} = \mu_0 \cdot I + \mu_0 \epsilon_0 \cdot \frac{d \Phi_E}{dt}$}
\end{equation}

Grunden til, at Maxwell udbygger Ampère's lov, er at loven kun er gældende for, at en stabil strøm er med til at danne en magnetisk flux. For at skabe symmetri med Faraday, udformede Maxwell sin teori om, at elektrisk flux også er gældende for at danne magnetisk flux ved en skiftende strøm. Derved er udtrykket $\mu_0 \epsilon_0 \cdot \frac{d \Phi_E}{dt}$ tilføjet til det oprindelige udtryk.

Maxwell's fire ligninger gør det muligt at beregne på trådløs energioverførsel, i forhold til trådløs mobilopladning.

\newpage