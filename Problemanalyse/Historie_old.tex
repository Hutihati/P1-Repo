\section{Elektromagnetismens historie}

Som baggrund for trådløs energioverførsel bragte fysikeren H. C. Ørsted (1777-1851) idéen om elektromagnetisme. \cite{historie2} H.C. Ørsted videreudviklede sine teorier gennem forsøg, hvilket førte til, at han i 1820 kunne bevise, at en magnetnål bliver påvirket af en elektrisk strøm. Hvis strømstyrken blev øget, blev kraften, der påvirkede magnetnålen større. H. C. Ørsted beskrev også, hvordan magnetfelter danner en lukket kreds. \cite{historie1}

En af de første der fokuserede på trådløs energioverførsel, gennem elektromagnetisme, og så det som en reel mulighed, var fysikeren Nikola Tesla (1856-1943). Tesla er en af opfinderne til radiokredsløb, og han har gennem sin forskning skabt basis for radio, TV og internet. \cite{historie3}

Nogle af de begreber, som Nikola Tesla byggede på var de opdagelser, som Michael Faraday (1791-1867) gjorde sig. Faraday var en af efterfølgerne til H. C. Ørsted, og han uddybede begrebet om, at magnetisme og elektricitet havde en tæt sammenhæng. I år 1831 beskrev Faraday princippet om elektromagnetisk induktion, hvor en elektrisk strøm kan blive skabt ud fra en skiftende magnetisk flux. \cite{historie4}

Hvor Faraday opdagede, hvordan magnetisme kunne inducere elektricitet, så beskrev den franske fysiker og matematiker André-Marie Ampère (1775-1836), hvordan elektricitet kan skabe magnetiske felter. Ampère var (ligesom Faraday) inspireret af H. C. Ørsteds tanker og forsøg, hvilket lå til basis for hans egne forsøg og formuleringer. En meget essentiel observation han foretog var, hvordan en ledning, der førte en strøm, kunne frastødes eller tiltrækkes af en anden ledning, som også førte en strøm. \cite{historie5} Ampere indså at denne påvirkning, de to ledninger havde på hinanden, måtte skyldes magnetisme, hvilket han senere formulerede igennem sin lov: Ampère's lov. \cite{historie6}

De teorier, som Faraday og Ampère udarbejdede, gav et billede på, hvordan magnetisme og elektricitet påvirker hinanden, men det var stadig kun beskrivende teorier. Det var den skottiske matematiker og fysiker James Clerk Maxwell(1831-1879), der gik dybere ned i teorierne, og beskrev fænomenerne med matematiske formler. Maxwell opstillede sine berymte fire formler (Maxwell's ligninger), som beskrev Gauss's lov for elektriske feltlinjer, Gauss's lov for magnetisme, Faraday's lov og Ampère's lov. Disse fire ligninger bliver brugt til beregning af trådløs energioverførsel. \cite{historie7}

\newpage