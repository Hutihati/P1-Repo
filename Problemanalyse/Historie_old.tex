\section{Historie??}

\textcolor{red}{DETTE ER GAMMEL HISTORIE}

H.C Ørsted var på ungdomsrejse i 1801, på denne rejse så han nogle billeder. Det var billeder, som tyskeren E.F.F. Chladni kunne fremkalde, disse billeder var avancerede geometriske mønstre. Chladni kunne fremkalde disse billeder, ved at han strøg en violinbue med forskellige toner imod en glasplade med sand på. Det som H.C Ørsted oprindeligt var interesseret i, var at han kunne gøre fint pulver elektrisk og derved skabe et mønster, dette kunne gøres ved at stryge violinbuen imod pladen. Dette eksperiment gjorde, at H.C Ørsted kunne finde en sammenhæng mellem elektrisk og mekanisk kraft. H.C Ørsted var inspireret af den tyske filosof Immanuel Kant. Immanuel Kant havde opstillet sin egen teori om naturen på trods, at han aldrig havde arbejdet med fysik. Immanuel Kants teori var, at der kun findes to naturkræfter, hvilket vil betyde, at elektricitet, magnetisme, varme og lys bare er de to kræfter, som er kombineret forskelligt. Magnetiske og elektriske kræfter måtte have en sammenhæng, det var H.C Ørsted overbevist om, dette skyldtes Immanuel Kant teori. 

Det var H.C Ørsted der i år 1820 opdagede, at en magnetnål med kræfter bliver påvirket af elektrisk strøm. Hvis strømstyrken i ledningen blev øget, betød det at kraften på magnetnålen blev større. H.C Ørsted opdagede, at et magnetfelt danner en lukket kreds. 

Det næste skridt blev dog ikke taget af H.C Ørsted, men af Michael Faraday. Michael Faraday som var udlært bogbinder, men senere blev han assistent for en berømt kemiker nemlig Humphry Davy. Michael Faraday opdagede i år 1831 princippet bag den elektriske tranformer og generator og elektromagnetisk induktion. Resten af årtiet arbejde Michael Faraday med, at udvikle sine teorier og ideer omkring elektricitet. 

Nikola Tesla var en amerikansk opfinder med serbiske rødder, han lavede offentlige demonstrationer, hvor han fremviste forsøg. Men Nikola Tesla fik aldrig rigtig den faglige anerkendelse, men han befandt sig i skyggen af Thomas Edison. Nikola Tesla var en af grundene til, at vekselstrøm vandt over jævnstrøm. Hvis der bliver nævnt radiokredsløb, skal der tænkes på Nikola Tesla, da han var den førende inden for dem, han forstærkede dem nemlig. Nikola Tesla gjorde ikke kun forarbejdet for radio og tv, men han lavede også forarbejdet for smartphones og ikke mindst internettet, dette gjorde han ved, at lave eksperimenter, som indeholdte et trådløst elektronisk netværk.

Nikola Tesla fremviste et forsøg i år 1891, som omhandlede trådløs strøm. Nikola Tesla stod med to gasudladningsrør også kaldet fluorescerende pære, blot i en tidligere udgave. Nikola Tesla tilsluttede ikke rørene til noget, men stod blot med dem i hånden, der var klemt nogle metalplader ind på scenen. Elektriciteten blev transmittet igennem luften, hvilket fik rørene til at lyse uden. Nikola Tesla ville gerne vide, om det var muligt at øge effekten, så det var muligt at kunne overføre trådløs strøm, over et større område.

Ud fra Nikola Tesla ideer, skabte han Tesla Tower. Nikola Tesla havde til hensigt, at han ville producere de første lyn-skala elektriske udladninger i menneskeheden. Nikola Tesla rejste en mast på 142-fod ovenpå sit laboratoriums tag, masten havde en kobber kugle i spidsen. Tårnets betydningsfulde ledninger blev ført igennem en utrolig stor højspænding Tesla-coil i laboratorium nedenunder masten.

\textcolor{red}{DETTE ER GAMMEL HISTORIE} 

\newpage