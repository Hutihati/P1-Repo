\section{Introduktion: Trådløs energioverførsel}

Noget der kommer lige efter indledning:

Hvor ser man wpt i dag?

Trådløs energioverførsel eller trådløs opladning er i dag implementeret i forskellige produkter bl.a. mellem en elektrisk tandbørste og dens opladerstation. Til selve energioverførslen benyttes induktiv kobling oftest, da det effektivt er muligt at overføre energien over en kort afstand. Induktiv kobling er en form af elektromagnetisme. Andre typer indenfor elektromagnetisme som mikrobølger kan også benyttes til trådløs energioverførsel, men da mikrobølger har en skadelig effekt på levende organismer, og induktiv kobling allerede bliver benyttet til formålet energioverførsel, gruppen tager derved et nærmere blik på denne type af elektromagnetisme.

Hvorfor er det relevant at kigge på wpt i dag?

Man kan altid spørge om, hvor vi ser teknologien i fremtiden, men et andet vigtigt perspektiv er, hvor relevant trådløs energioverførsel er i dag. Vi er et progressivt folkefærd, og vi vil altid gerne anerkendes for vores projekter. Vi er nået så langt, som vi kan nå med ledninger, så det næste skridt må være at fjerne ledningerne. Med den store ændring i regeringernes klimaaftaler, skal vi også tænke på transport. Vi bliver nødt til at finde en måde at få vores elbiler til at køre længere, en måde at kunne gøre det kan være ved at forbedre batterierne eller lade bilen lade op, imens den kører. Her kan man også snakke om busser, som når de holder og samler passagerer op, kan nå at lade lidt op inden den kører videre. Ved at busserne hele tiden kører rundt, er de en af de store syndere for CO2, og det kunne være en løsning på at skære ned på CO2-udledningen, vi har i dag.

Hvor ser vi at vi kan bruge det henne?

Teknologien giver base for en række nye muligheder for brug af elektriske produkter, ikke kun som gadgets, men også indenfor transportmidler og større maskineri på fabrikker. Fremadrettet ville der skabes mulighed for ét samlet energisystem, der kan registrere, hvis ens elektriske produkter mangler strøm. Herfra kan der blive overført strøm fra en transmitter til ens elektriske produkter, uden man skal tilslutte dem en ledning i en stikkontakt. Dette kunne bl.a. implementeres i ens hjem, hvor transmittere ville kunne oplade din mobil ligegyldigt, hvor i huset man befinder sig, samtidig med at det ville kunne drive køleskabet i køkkenet og tv'et i stuen.

Set i et andet perspektiv, ville teknologien også gøre elektriske transportmidler som biler og busser mere attraktivt. Hvis transmittere bliver implementeret i vejnettet, ville man kunne oplade sin bil, mens man kører. Det vil også kunne spille godt sammen med Elon Musks planer for, at have Tesla til at lave busser, som kan skifte benzin buerne ud.

En anden ting kunne være, at man ikke længere skulle trække store mængder kabler gennem fabrikshaller eller bare den almindelige husstand. Dette ville skabe en mere mobiliseret produktion, hvor fabrikkerne ikke skal tage højde for, hvordan maskinerne skal placeres. Derved er der strøm til hele produktionen uden at ledninger og kabler løber på tværs af fabrikshallen.

\newpage