\section{Introduktion: Trådløs energioverførsel}

Trådløs energioverførsel eller trådløs opladning er i dag implementeret i forskellige produkter bl.a. mellem en elektrisk tandbørste og dens opladerstation. Til selve energioverførslen benyttes induktiv kobling oftest, da det er muligt, at kunne overføre enerigen effektivt over en kort afstand. Induktiv kobling er en form for elektromagnetisme. Andre typer indenfor elektromagnetisme som mikrobølger kan også benyttes til trådløs energioverførsel. Men da det viser sig at induktiv kobling giver den bedste og mest effektive overførsel \cite{mit}, så har gruppen valgt at fokusere rapporten på denne form for kobling, altså induktiv kobling.

Det kan være relevant at spørge om, hvor teknologien måske kan ses i fremtiden. Et andet vigtigt perspektiv er, hvor relevant trådløs energioverførsel er i dag. Danmark er et progressivt folkefærd, og vil altid gerne anerkendes for vores projekter, specielt indenfor grøn og bæredygtig energi. Danmark er nået så langt med ledningsteknologien at udviklingen på dette punkt er haltet, altså der er ikke er sket større teknologiske fremskridt længe. Det næste logiske skridt må være at undersøge mulighederne for at fjerne ledningerne. I denne anledning spare ressourcer, både i råmaterialer som kobber og gummi, men også mande timer, til at grave, el. konstruere tårne ledningerne kan hænge mellem. Dette vil i sidste ende spare penge og miljøet hvis den trådløse energiteknologi opnår en brugbar effektivitet over længere afstande.

Med den store ændring i regeringernes klimaaftaler, kan det være relevant også at tænke på transportsektoren.

Noget der har vist sig at gavne miljøet er udnyttelsen af elbiler. Denne form for transport står med problemstillingen, at deres mulige kørselsafstand før ladning ikke kan måle sig med benzinkøretøjers afstand før tankning. Sammenlign Tesla Motors Model S elbil kørselsafstand på ca. 500  km, som er højt sat ved normal kørsel \cite{tesla}, sammenlign det så med en benzinbil med en normal 60 L tank der kører 20 km/L som i sidste ende er en afstand på  1200 km. Dette er ikke et unormalt forbrug ved normal kørsel i moderne biler. Det ses at disse tal stadig er langt fra hinanden selv med de store forbedringer firmaer som Tesla Motors har spearheaded i den senere tid.
Et løsningsforslag til dette kunne være at lade bilerne mens de var på vejen. En stor synder på vejene er de konstant kørende busser, hvis det skal være muligt at få dem på el, og få dem til at lade mens de var på vejen, ville det effektivisere dem og gøre dem betydeligt mere miljøvenlige. Altså hvis energi senderne kunne blive implementeret i vejnettet, ville det være muligt at oplade køretøjer, mens man kører på vejen. Det vil også kunne spille godt sammen med Elon Musks planer for at sætte elbusser på vejene \cite{tesla}. Der er selvfølgelig en lang liste med problematikker for at implementere disse løsninger med trådløs energioverførsel. Nogen af disse problematikker vil blive undersøgt i denne rapport.

Teknologien giver base for en række nye muligheder for brug af elektriske produkter, ikke kun som gadgets, men også indenfor transportmidler og større maskineri på fabrikker. Fremadrettet ville der skabes mulighed for ét samlet energisystem, der kan registrere, hvis ens elektriske produkter mangler strøm. Herfra kan der blive overført strøm fra en sender station til ens elektriske produkter, uden man skal tilslutte dem en ledning i en stikkontakt. Dette kunne bl.a. implementeres i ens hjem, hvor transmittere ville kunne oplade din mobil ligegyldigt, hvor i huset man befinder sig, samtidig med at det ville kunne drive køleskabet i køkkenet og tv'et i stuen.

En anden ting kunne være, at man ikke længere skulle trække store mængder kabler gennem fabrikshaller. Dette ville skabe en mere mobiliseret produktion, hvor fabrikkerne ikke skal tage højde for, hvordan maskinerne skal placeres, med den viden at det bliver et stort økonomisk tab at bruge tid og energi på at flytte dem. Derved er der strøm til hele produktionen uden at ledninger og kabler løber på tværs af fabrikshallen, og muligt at flytte og omrokere dem hurtigt og effektivt.\cite{fabrik}

\newpage