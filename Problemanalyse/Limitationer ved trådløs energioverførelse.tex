Limitationer ved trådløs energioverførelse 
trådløs energioverførelse er en teknologi med mange mulighed og en fremtid for at blive vider undersøgt og  videreudviklet. Med projektets fokus på mobilopladning vil der i dette afsnit kigges på hvilke limitationer og ulemper der er ved induktiv energioverførelse til henholdsvis opladning af mobiltelefoner. fokusset vil ligge størst på selve energioverførelse i forhold til effekten og distancen der kan overføres med, hvor høj frekvenser kan der bruges og hvilken betydning har det, og til sidst hvor energi kan der oplades med. Der vil være et mindre fokus på varmeudviklingen i spolerne og der lovmæssige aspekter der kan begrænsning.

Effekten af opladning afhænger af en del faktor, distancen fra spolerne, hvor høj frekvens, hvor stærk strøm. Med i dags teknologi kan der opnås en effekt på 90 procent og bedre indenfor en distance på 10cm. Dette vil sige at indfor 10cm vil der kun være et tab på 10 procent, der går til varme og ud i luften. Størrelsen af spolerne og frekvensen influere også til størrelsen af den magnetiske flux der dannes ved spolerne.